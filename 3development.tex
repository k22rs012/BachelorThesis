% 開発環境


\section{概要}
本章では,本研究で使用した開発環境,測定機器に搭載するモジュールについて記述する.

\section{Arduino IDE}  
測定機器とマイクロコントローラの制御はArduino IDEで行う.Arduinoの開発環境であるArduino IDE (Integrated Development Environment)はArduinoボード上で動作するソフトウェアを開発するための統合開発環境である.
ソースコードの編集,コンパイル,デバッグ,およびシリアルモニタ(コンソール機能)を用いたデータ送受信機能などが提供される.
またArduino IDEには,様々なArduinoボードや他社のマイクロコントローラに対応するためのボードパッケージマネージャや各種 I/O(センサ等を含む) サポートするライブラリマネージャ機能があり,ライブラリが豊富に提供されているため,CO2センサやその他のモジュールのセットアップが迅速に行える.
また,今回使用したArduinoIDEのバージョンは2.3.6(Date: 2025-04-09T11:22:51.016Z)である.


\section{測定機器構成}
図\ref{fig:sensors}に測定機器に搭載するモジュールを示す.
本測定機器は
マイクロコントローラ、CO2センサで構成されている.
CO2センサモジュールはHailege SCD41を使用した.


\section{測定機器に搭載するモジュール}
本研究では,CO2センサからのデータ取得や処理を行うために,マイクロコントローラを使用している.
選定したマイクロコントローラはSeeed Studio XIAO ESP32C6である。
以下に詳細を述べる.


\section{マイクロコントローラ}
マイクロコントローラとは,RAM,ROM,プロセッサ,I/O ポートなどを単一の集
積回路(IC)にまとめた,組み込み用マイクロプロセッサのことである.マイクロコント
ローラSeeed Studio XIAO ESP32C6(図\ref{fig:SeeedESP32C6})を使用する.


\subsection{Seeed Studio XIAO ESP32C6}  
ESP32-C6 は,Espressif Systems 社が開発したWi-Fi 6 対応のマイクロコントローラであり,2.4 GHz 帯 Wi-Fi および Bluetooth Low Energy(BLE)を内蔵している。
本研究では,無線通信機能を外付けすることなく測定データをサーバへ送信できる点に加え,バッテリ駆動を想定した 3.3 V ピンを備え,小型デバイスへの組込みに適している点を評価し,ESP32-C6 を採用した。

ESP32-C6 は 32 ビット RISC-V プロセッサを搭載し,十分な演算性能と低消費電力動作を両立している。また,小型パッケージで多数の GPIO やSPI,UART,I\textsuperscript{2}C などの周辺インタフェースを備えており,センサとの接続や携帯型デバイスへの組込みに適している。



\begin{figure}[h]
\centering
\includegraphics[width=.45\linewidth]{./figures/SeeedESP32C6.eps}
\caption{SeeedStudioXIAOESP32C6}
\label{fig:SeeedESP32C6}
\end{figure}



\section{センサ}
本研究では,CO2濃度の測定にHailege社のSCD41 CO2 二酸化炭素ガスセンサ モジュールとTassety社のSCD40 Gas Sensor Moduleを使用する。センサにはそれぞれSensirion社のCO2センサSCD40、SCD41を搭載している。以下に主な特性を述べる。

\section{二酸化炭素センサ SCD41}

二酸化炭素濃度を測定するセンサとして,Sensirion社のSCD41(図\ref{fig:SparkfunFS3000-1015})を使用した.
SCD41は,光音響分光法(Photoacoustic Spectroscopy)を採用した高精度なCO$_2$センサであり,小型でありながら高精度な計測が可能である.
この方式は,CO$_2$分子が特定の赤外線波長を吸収した際に発生する音波を検出することで濃度を算出するものであり,従来のNDIR方式に比べて光学経路を短縮できるという利点を有する.

また,SCD41は温度および湿度センサを内蔵しており,環境補正を自動的に行うことで安定した測定値を得ることができる.
測定範囲は$400 \sim 5,000$\,ppm,精度は$\pm(40\,\mathrm{ppm}+5\%)$である.
動作電圧は$2.4 \sim 5.5$\,Vと広く,低消費電力設計が施されているため,バッテリ駆動の組込み機器にも適している.

さらに,SCD41はI$^2$C通信を介してマイクロコントローラと接続され,取得したCO$_2$濃度・温度・湿度データをリアルタイムに処理することが可能である.
本研究では,省電力化のためにSCD41のシングルショット測定モードを用い,必要時のみ測定を行う構成とした.
SCD41の主な仕様を図\ref{fig:fs3000-1015Parameters}に示す.




\begin{table}[htbp]
\centering
\caption{SCD41の主なパラメータ}
\begin{tabular}{|c|c|}
\hline
\multicolumn{1}{|c|}{モデルNo.} & \multicolumn{1}{|c|}{SCD41} \\ \hline \hline
I2Cアドレス             & 0x62 \\ \hline
測定対象               & CO$_2$,温度,湿度 \\ \hline
CO$_2$測定範囲         & 400$\sim$5000 ppm \\ \hline
CO$_2$測定精度         & $\pm$(40 ppm + 5\%) \\ \hline
温度測定範囲           & -10$\sim$60 ℃ \\ \hline
湿度測定範囲           & 0$\sim$95 \%RH \\ \hline
解像度                 & 16ビット \\ \hline
入力電圧               & 2.4$\sim$5.5 V \\ \hline
平均消費電流           & 約15 mA \\ \hline
ボード寸法             & 約10.1 mm $\times$ 10.1 mm $\times$ 7.0 mm \\ \hline
対応温度               & -10$\sim$60 ℃ \\ \hline
\end{tabular}
\label{tab:scd41Parameters}
\end{table}


% \subsection{MEMS}
% MEMS(Micro-Electro-Mechanical Systems)は,マイクロメートル(μm)スケールの微細構造を持つデバイスで,電気的,機械的,さらには化学的な機能を統合的に備えたシステムを指す.これらのデバイスは,半導体製造技術や微細加工技術を応用して製造されるため,高い精度と小型化が可能である.MEMS技術は,主にセンサ,アクチュエータ,およびこれらを統合したシステムとして応用され,幅広い分野で重要な役割を果たしている.


% \subsection{サーモパイル}
% サーモパイル(Thermopile)は,複数の熱電対(Thermocouple)を直列に接続して構成されたデバイスで,温度差を電圧に変換する機能を持つ.このデバイスは,熱電効果(ゼーベック効果)を利用して動作し,温度差に比例した電圧を発生させる.サーモパイルは接触型および非接触型温度測定,放射エネルギーの検出,さらにはエネルギー収集(エナジーハーベスティング)などに広く利用されている.



\section{通信プロトコル}
測定機器に搭載したモジュール間の通信にはI2C を使用している.
I2C はシリアル通信の一部で,
シリアル通信とはデータを送受信するための信号線を1 本または2 本使用して,
データを1 ビットずつ連続的に送受信する通信方式である.
シリアル通信以外にも,データを送受信するための信号線を1 本または2 本使用して,
データを同時に送信するため複数ビットで送受信するパラレル通信がある.
以下にシリアル通信方式の,I2C,UARTそれぞれの特徴を述べる.



\subsection{I2C通信}  
I2C 通信とは,I2C(Inter-Integrated Circuit)の略称である.
I2C 通信はフィリップス社が提唱する通信インターフェースでクロックに同期させてデータの通信を行う同期式シリアル通信の一つである.I2C ではクロックを送るための端子(SCL),
データ入出力する端子(SDA)の2 本の信号線を用いて通信を行い,
マイコンとマイコン周辺機器の通信に用いられることが多い.
I2C 通信では,マスタとスレーブという二つの役割に分け,
マスタからスレーブに対して送信や受信の指示を行う.
クロックはマスタから出力され,入力と出力はクロックに同期して行われる.
1 つのマスタで複数のスレーブと通信することが可能なため,
省配線や省スペース化が実現可能である.
I2C 通信がSPI 通信と大きく異なる点として,
個々のスレーブがアドレスを持っておりデータの中にアドレスが含まれていること,
1バイト転送毎に受信側からACK 信号(確認応答) をして互いに確認を取りながらデータ転送を行っていることが上げられる.


\subsection{UART}  
UART通信とは,UART(Universal Asynchronous Receiver Transmitter)の略称で,
クロックの信号線が存在せず,
データ送信用の端子(TX)とデータ受信用の端子(RX)2 本の信号線で通信を行う非同期式シリアル通信の一つである.
コンピュータと外部の機器を繋ぐシリアルポートやシリアルインターフェースの制御に用いられることが多い.
UARTではクロックの信号線が存在しないため,
通信をする前にあらかじめ送受信するデバイス間で通信速度(ボーレート)を決めておく必要がある.
ボーレートとは1 秒間に何bitのデータを転送するかを意味し,bps(bit per second)という単位で表記される.
UARTでよく使用される通信速度は4,800,9,600,19,200,38,400,57,600,115,200bps である.



\section{AVHzYct-3}

今回,測定機器の消費電力を測定する際に使用するツールとして,AVHzYct-3を用いた.
図\ref{fig:AVHzYct-3}にAVHzYct-3を示す.
AVHzYct-3は,機器とUSBで接続することにより,その機器の電圧や電流,消費電力,対応する急速充電規格を確認することができる.
また,パソコンに接続し,専用のソフトを使用することで,詳細なデータを取得し,保存することができる.
実際にパソコンで表示したデータは図\ref{fig:AVHzYct-3Reco}に示す.
AVHzYct-3を使用することにより,対象機器の消費電力や,時間毎の電圧,電流,経過時間,極値などを確認することができる.
波形データからは,測定機器がどのタイミングで電力を消費しているかを確認でき,消費電力のピーク値や消費電力の変化を確認することができる.
また,消費電力の時間的変化からスリープ状態とデータの取得,送信時の遷移が確認でき,測定機器が安定して計測・データ送信を行っているかどうかも確認することができる.

\begin{figure}
\centering
\includegraphics[width=0.6\linewidth]{figures/AVHzYct-3.eps}
\caption{AVHzYct-3}
\label{fig:AVHzYct-3}
\end{figure}

\begin{figure}
\centering
\includegraphics[width=0.8\linewidth]{figures/AVHzYct-3Reco.eps}
\caption{AVHzYct-3のデータ取得画面}
\label{fig:AVHzYct-3Reco}
\end{figure}



\section{サーバ}
サーバ側の処理は AlmaLinux 9.5 で動作している MariaDB を利用している.
MariaDB はMySQL から派生したオープンソースの RDBMS(リレーショナルデータベース管理システム)である.
MariaDBは,高速で信頼性が高く,複数のデータベースを同時に操作することができるため,センサデータの取得や処理に適している
MySQL と高い互換性を維持しつつセキュリティやパフォーマンスが向上しているという利点がある.
今回は PHP を使用して MariaDB に接続し,JavaScriptでデータを処理,ブラウザに表示している.



\subsection{PHP}
PHPは,Webアプリケーションの開発に広く使用されているスクリプト言語である.
本研究では,PHPを使用して,センサデータの収集,処理,および可視化を行っている.
PHPは,データベースとの連携が容易であり,Webサーバー上で動作するため,センサデータをリアルタイムで取得し,Webブラウザに表示することが可能である.



\subsection{JavaScript}
JavaScriptは,Webページの動的なコンテンツを制御するためのスクリプト言語である.
本研究では,JavaScriptを使用して,Webページ上でのセンサデータのリアルタイム表示やグラフの描画を行っている.
JavaScriptは,Webブラウザ上で動作するため,センサデータをリアルタイムで取得し,Webページ上に
表示することが可能である.



\section{Blender}
Blenderは,オープンソースの3Dコンピュータグラフィックスソフトウェアである.
本研究では,測定機器用のケースを作成するために使用した.Blenderを使用して,測定機器用のケースの3Dモデルを設計し,3Dプリンタで出力している.
Blenderは,高度な3Dモデリングやアニメーションを行うことができるため,設置台のデザインや機能の検証に適している.
今回3Dモデルの印刷には本大学にあるオープンイノベーションセンタの3Dプリンタを使用している.
