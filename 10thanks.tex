本研究を執筆するにあたり,熱心なご指導を賜り,
また電子情報通信学会などの活動を通じて多大なるご支援をいただいた九州産業大学理工学部情報科学科の田中康一郎教授に,
心より感謝を申し上げます.先生の的確な助言と温かいご指導により,研究を進める上で数々の困難を乗り越えることができました.

ならびに,研究の協力や指導を惜しみなくしてくださり,卒業論文の執筆においても多大なるご助力をいただいた修士2年生の安部萌氏には,深く感謝の意を表します.
また,2年間にわたり共に研究に取り組み,時には助言を与え,時には支え合いながら多くの成果を上げることができた学部4年生の佐藤一晴氏,
千原泰護氏,友藤沙南氏,林智也氏,渡邊和幸氏にも,心より感謝いたします.皆様の励ましやご協力がなければ,
本研究を完成させることは難しかったと感じております.今後ますますのご健勝とご活躍をお祈り申し上げます.

さらに,学部3年生の青柳貴之氏,井上誠斗氏,上原一眞氏,木村光翔氏,須賀翔一氏,竹内勇人氏,平田健斗氏,
嶺田あんず氏につきましては,研究活動を手伝っていただいたことに深く感謝申し上げます.皆様の協力により,
研究活動がより円滑に進められ,多くの成果を得ることができました.今後のさらなる成長とご活躍を心よりお祈り申し上げます.

最後に,大学生活の4年間において,多方面から物心両面で支えてくださった家族に,改めて心より感謝申し上げます.
家族の支えがあったからこそ,困難を乗り越え,研究や学業に専念することができました.
この場を借りて深い感謝の意を表し,今後も恩に報いられるよう努力を重ねてまいります.

\begin{flushright}
2025/1

水野良基
\end{flushright}