% 測定機器実装


\begin{comment}
二酸化炭素濃度測定器の選定等に関するガイドライン」を参考にしながら、要件仕様を満たす測定機器の開発を行なった。
作成した測定機器は以下の通りである。

測定機器を作成するにあたり、最初にESP32-C6とSCD41を組み合わせたプロトタイプを作成した。そこで動作の確認ができた為、ボタンとLEDを取り付け測定機器1を作成した

測定機器1はボタンを押すことでDeepSleepから起動し、CO2濃度を測定したのちにデータをサーバに送信し、再びDeepSleepに戻る。測定した結果やデータが正しく送信できたかなどがわかるようにLEDの照明で場合分けを行っている。DeepSleepについては後ほど説明する。

測定機器2は測定機器1の縮小版である。ESP32-C6の裏面にも基板を使用して前と後ろを使用して作成することにより、測定機器1よりもサイズが三分の一になった。サイズが三分の一になったことにより、より持ち運びがしやすくなった。


測定機器3はLTEを使用して通信できるようにしている。測定機器2まではwifiを使用しているので屋外に出る際はBluetoothを使用して登録をしているwifiのssidとpasswardを使用して、スマートホンのテザリングを用い測定データを取得する必要があった。その為、お年寄りやお子様方には操作が難しかったり、そもそも使用者に面倒な作業をさせてしまう問題点があった。そのためこの測定機器3ではSIM7080Gを使用して、LTEで通信することができる。そのため屋外でも、屋内でも常に端末一つで通信が可能であり、面倒な作業もなくなった。
\end{comment}



\section{測定機器の開発方針}
本研究では,「二酸化炭素濃度測定器の選定等に関するガイドライン」を参考にしながら,要件仕様を満たす CO$_2$ 測定機器の開発を行った。開発した測定機器は,通信方式や筐体構成の異なる複数の試作機から構成されており,段階的な改良を通して実用性の向上を図っている。

\section{測定機器1の設計と実装}
測定機器の開発にあたり,最初に ESP32-C6 と SCD41 を組み合わせたプロトタイプを作成した。このプロトタイプにより,CO$_2$ 濃度の測定およびサーバへのデータ送信が正常に行えることを確認した。これらの動作確認を踏まえ,ボタンおよび LED を搭載した測定機器1を作成した。測定機器1は,ボタンを押すことでディープスリープ状態から起動し,CO$_2$ 濃度を測定した後,測定データをサーバへ送信し,再びディープスリープ状態へ移行する構成とした。
また,測定結果やデータ送信の成否を視覚的に確認できるよう,LED の点灯パターンによる状態表示を行っている。ディープスリープ機能については,後述する。

\section{測定機器2の設計と実装}
測定機器2は,測定機器1の機能を維持したまま,筐体の小型化を目的として設計したものである。ESP32-C6 の表面および裏面の両方に基板を配置する構成とすることで,部品配置の高密度化を行い,測定機器1と比較してサイズを約 3 分の 1 に縮小した。この小型化により,携帯性が向上し,日常生活において持ち運びやすい測定機器となった。

\section{測定機器2の課題}
一方で,測定機器2は Wi-Fi 通信を用いてデータ送信を行うため,屋外で使用する場合には,スマートフォンのテザリング機能を利用する必要があった。具体的には,Bluetooth を用いてWi-Fi の SSID およびパスワードを登録し,スマートフォン経由で通信を行う構成である。この方式は,利用者に対して事前設定や操作を要求するため,高齢者や子どもにとっては操作が難しい場合がある。また,利用者に追加の操作負担を与えてしまう点が課題として挙げられた。

\section{LTE通信方式の検討}
これらの課題を解決するため,Wi-Fi に依存しない通信方式としてセルラー通信の導入を検討した。屋内外を問わず安定した通信を行うことを目的として,LTE 通信を用いた測定機器の開発を行った。

\section{測定機器3の設計と実装}

測定機器3では,測定機器2で課題となった通信方式の問題を解決するため,LTE 通信モジュールである SIM7080G を搭載した。これにより,Wi-Fi 環境やスマートフォンのテザリングに依存せず,屋内外を問わず単体でサーバへのデータ送信が可能となった。

本構成により,利用者が通信設定を行う必要がなくなり,操作負担を大幅に軽減できる。その結果,高齢者や子どもを含む幅広い利用者にとって扱いやすい CO$_2$ 測定機器の実現が可能となった。

表\ref{tab:device_comparison} に,本研究で開発した測定機器1~3の構成および特徴を示す。測定機器1から測定機器3へと段階的に改良を行うことで,携帯性および実用性の向上を図った。
\begin{table}[htbp]
\centering
\caption{開発した測定機器の比較}
\begin{tabular}{|c|c|c|c|}
\hline
項目 & 測定機器1 & 測定機器2 & 測定機器3 \\ \hline \hline
マイクロコントローラ & ESP32-C6 & ESP32-C6 & ESP32-C6 \\ \hline
CO$_2$ センサ & SCD41 & SCD41 & SCD41 \\ \hline
通信方式 & Wi-Fi & Wi-Fi & LTE(SIM7080G) \\ \hline
屋外利用 & テザリングが必要 & テザリングが必要 & 単体で可能 \\ \hline
電源方式 & Li-Po バッテリ & Li-Po バッテリ & Li-Po バッテリ \\ \hline
操作方法 & ボタン操作 & ボタン操作 & ボタン操作 \\ \hline
状態表示 & LED 表示 & LED 表示 & LED 表示 \\ \hline
筐体サイズ & 標準 & 約1/3に小型化 & 測定機器2と同等 \\ \hline
主な目的 & 動作確認 & 携帯性向上 & 実用性向上 \\ \hline
\end{tabular}
\label{tab:device_comparison}
\end{table}


\section{DeepSleep を用いた省電力制御}

本研究で開発した CO$_2$ 測定デバイスは,携帯型デバイスとして長時間動作することが求められるため,消費電力の低減が重要な設計要件となる。そこで,本研究では ESP32-C6 が備えるDeepSleep 機能を用いた省電力制御を採用した。DeepSleep とは,マイクロコントローラの動作を一時的に停止し,必要最小限の回路のみを動作させる低消費電力モードである。この状態では,CPU や無線通信機能などが停止されるため,通常動作時と比較して消費電力を大幅に削減することが可能である。

測定機器1〜3では,通常時は DeepSleep 状態とし,ユーザがボタンを押した際にDeepSleep 状態から復帰する構成とした。復帰後は CO$_2$ 濃度の測定を行い,測定データをサーバへ送信した後,再び DeepSleep 状態へ移行する。このような動作により,測定および通信を行う時間を最小限に抑え,バッテリ消費を抑制している。以上の構成により,リチウムポリマーバッテリによる駆動においても,携帯型 CO$_2$ 測定デバイスとして実用的な動作時間を確保することが可能となった。

\section{LTE 通信を用いたデータ送信}

測定機器3では,セルラー通信モジュールである SIM7080G を用いてLTE 回線によるデータ通信を行っている。SIM7080G は,LTE 回線を利用したデータ通信が可能なモジュールであり,広い通信エリアを有することから,屋内外を問わず安定した通信を行うことができる。

測定機器3では,ESP32-C6 と SIM7080G を接続し,測定した CO$_2$ 濃度データをLTE 回線を介してサーバへ送信する構成とした。この方式により,Wi-Fi の SSID やパスワードの設定,スマートフォンによるテザリング操作が不要となり,測定機器単体で通信が可能となっている。以上の構成により,測定環境や利用者の通信環境に依存せず,安定した CO$_2$ 濃度データの取得が可能となった。