% 結論

\section{まとめ}


\begin{comment}
一つ目は,空調に設置可能で,自動で風速や温湿度を測定できるIoT機器の開発である.
特に,風速については,405iの測定値と比較して誤差を10\%以内に抑えることを目標とし,
さらに1週間の稼働を目指す.
二つ目は,測定機器の測定値を記録し,測定値をリアルタイムで閲覧できるシステムの開発である.
\end{comment}


近年の温暖化や異常気象による急激な温度差の変化により,
空調設備の需要が高まる中,空調設備の保守点検作業の効率化が求められている.
現在の空調設備の保守点検作業は,手動で測定を行うため,作業員の負担が大きい上に,作業員ごとに測定値が異なることがある.
そこで本研究プロジェクトでは風速や気温,湿度を自動で測定し,測定値を閲覧できるシステムを開発することで,
空調設備の保守点検作業を効率化を目標とした.

この目標を達成するために,以下の段階を踏んで開発を進めていく.
風速や温湿度を測定できる機器を開発し,
自動で測定を行えるようなシステムを開発する.
そして,無線で測定値を送信でき,測定値を閲覧できるシステムを開発する.
次に,測定機器が1週間程度の長期間にわたって稼働できるように設計し,
最終的に測定機器の小型化やシステムの低コスト化を目指す.

本研究の目的として,以下の二つを掲げた.
一つ目は,測定機器については405iの測定値と比較して誤差を10\%以内に抑えること,および1週間の稼働を達成することを目標とした.
二つ目は,閲覧システムについては測定後20秒以内にデータを閲覧可能とすることを目標とした.
これらの目標を基に,ハードウェアとソフトウェアの両面から開発を進めた.

測定機器は吹出し口用の測定機器を4つ,吸込み口用の測定機器を3つ開発し,
それぞれの測定機器の風速の測定精度,稼働時間,データの測定と送信の安定性を評価した.

風速の測定精度については,吹出し口用の測定機器の風速の測定精度を評価するため,405iと比較して風速を測定した.
結果として,吹出し口タイプ3の測定が405iと比較して誤差を1.35\%に抑えることができ,高い精度での測定が可能であることが確認された.
稼働時間については,2400mAhで122時間の稼働が確認できたが,理論値よりも30時間短い結果となった.
稼働時間の改善については,今後の課題として処理の効率化や省エネ設計をさらに進める必要がある.

作成した測定機器の中で不安定な動作を示す測定機器も確認できたが,構造によるものかセンサの不良によるものかを調査することで,今後の改善につなげることができる.
基板を使用していない吸込み口タイプ3はディープスリープを適用する前はマイコンが発生する熱が籠ったままで正しい測定が行えなかったが,
ディープスリープを適用することで正常に測定を行うことが可能となった.
これは,ディープスリープからの復帰で起こる負荷よりも,WiFiやセンサの測定で起こる負荷の方が大きいことが原因であると考えられる.

閲覧システムについては,測定後20秒以内に測定されたデータを閲覧するシステムを開発できた
また,測定機器の測定間隔を調整することで,さらに即時性の高いデータの閲覧も可能である.


以上の結果から,風速,温度,湿度を自動で測定し,風速の測定誤差を10\%以内に収めるとともに,測定値をリアルタイムで閲覧可能なシステムの開発に成功した.
さらに,既存のサービスと比較しても引けを取らない性能を有しており,空調設備の保守点検作業の効率化に貢献できると考えられる.




\section{今後の課題}
今後の課題としては,以下の点が挙げられる.

\subsection{稼働時間の向上}
測定間隔を20秒のまま維持しながら,測定機器の稼働時間を1週間に延長することである.
これには,現在の20秒ごとに消費電力の大きいWiFiに接続する処理ではなく,
データを一時的に保存し,一定時間ごとにまとめて送信することで,WiFiに接続する回数を減らし,消費電力を削減するという方法が考えられる.
また,WiFiの接続に失敗した際の消費電力が大きいため,安定して接続できるように接続するWiFiルータの設定や設置位置を見直すことも必要である.


\subsection{測定機器の小型化}
測定機器の小型化を実現することである.
小型化により,空調設備への設置方法が多様化し,現在測定に使用しているカセット型の空調機器以外にも設置できるようになる可能性があり,
より幅広い用途での測定を可能にできると考えられる.


\subsection{システム全体のコストダウン}
システムの構成や測定機器の開発手順を見直すことで,より運用コストの低いシステムを構築することができる.
具体的には,現在コンパイル時にのみWiFiの設定が可能なため,特定の条件の時に,BlueToothなどの接続手段を使用し,
WiFiの設定を行うことで,運用コストを削減することができる.

