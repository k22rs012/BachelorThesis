% 結論

\section{まとめ}


\begin{comment}
一つ目は,空調に設置可能で,自動で風速や温湿度を測定できるIoT機器の開発である.
特に,風速については,405iの測定値と比較して誤差を10\%以内に抑えることを目標とし,
さらに1週間の稼働を目指す.
二つ目は,測定機器の測定値を記録し,測定値をリアルタイムで閲覧できるシステムの開発である.
\end{comment}



本研究では,日常生活環境における換気状態の把握を目的として,携帯可能な小型 CO$_2$ 測定デバイスの設計・試作および評価を行った.従来の据え置き型 CO$_2$ 測定器は,設置場所が限定されることや,利用者の周囲環境を直接的に把握することが難しいという課題があった.これに対し,本研究では,利用者が身につけて使用できる携帯型デバイスに着目し,小型化・省電力化・測定精度・通信機能を両立したCO$_2$ 測定システムの実現を目指した.

まず,設計段階においては,経済産業省および産業用ガス検知警報器工業会により制定された「二酸化炭素濃度測定器の選定等に関するガイドライン」を参考とし,CO$_2$ 濃度測定器に求められる基本的な性能を満たすことを重視した.具体的には,呼気中の CO$_2$ に対して適切に応答する一方で,消毒用アルコールなどの揮発性成分に対して誤反応しないこと,および安定した測定が可能であることを設計方針として採用した.これらの指針に基づき,光学式 CO$_2$ センサを中心とした構成を採用し,携帯型デバイスとしての実用性を考慮したシステム設計を行った.

次に,試作した携帯型 CO$_2$ 測定デバイスについて,実環境での測定に先立ち,センサの基礎特性確認を実施した.アルコールおよび呼気に対する応答を評価した結果,アルコールに対しては測定値の大きな変動が見られなかった一方で,呼気を吹きかけた際には CO$_2$ 濃度の顕著な上昇が確認された.この結果から,本研究で使用した CO$_2$ 測定デバイスは,ガイドラインに示される基本的な応答特性を満たしており,実環境における CO$_2$ 濃度測定に適したセンサであることが確認された.

据え置き型 CO$_2$ センサとの比較測定では,同一環境下において両者を設置し,長時間にわたる CO$_2$ 濃度の時間変化を比較した.その結果,測定開始直後に見られた高い CO$_2$ 濃度は,設置時に呼気がセンサにかかった影響であると考えられるものの,その後の時間変化においては,据え置き型センサと携帯型センサが同様の増減傾向を示すことが確認された.一部の時間帯において測定値に差が見られたものの,CO$_2$ 濃度の上昇および低下のタイミングは概ね一致しており,携帯型デバイスが環境中の CO$_2$ 濃度変化を適切に捉えていることが示された.

さらに,赤村ドームハウスにおける測定では,高さ方向に CO$_2$ 濃度の分布が生じる環境において,携帯型デバイスが利用者周囲の CO$_2$ 濃度を把握できるかを検証した.多数の据え置き型センサによる測定結果と比較した結果,携帯型デバイスによる測定値は,設置位置に応じた CO$_2$ 濃度の違いを反映しており,空間内の換気状態を把握する手段として有効であることが確認された.この結果から,携帯型 CO$_2$ 測定デバイスは,空間全体の分布を把握する用途に加え,利用者の周囲環境を評価する用途においても有用であると考えられる.

また,電車内における測定では,公共交通機関という人の乗降や換気条件が刻々と変化する環境において測定を行った.ドア付近およびドアから離れた座席位置において測定した結果,ドア開閉や乗客の乗降に伴う CO$_2$ 濃度の変化が確認され,測定位置による濃度差が生じることが示された.このことから,携帯型 CO$_2$ 測定デバイスは,移動環境においても測定可能であり,公共交通機関内の換気状況や空気環境を評価する手段として活用できる可能性が示唆された.

省電力性能の評価においては,周期的な測定・通信・待機動作を採用することで,バッテリ駆動による長時間動作が可能であることを確認した.これにより,携帯型デバイスとして日常的に使用する際にも,頻繁な充電を必要としない実用的な運用が可能であると考えられる.

以上の結果から,本研究で試作した携帯型 CO$_2$ 測定デバイスは,ガイドラインに基づく基本的な測定性能を満たし,据え置き型センサと同様の CO$_2$ 濃度変化を捉えることができること,ならびに屋内外や移動環境を含むさまざまな実環境において利用可能であることが示された.本研究の成果は,換気状態の可視化や空気環境の把握を目的とした携帯型 CO$_2$ 測定システムの実現に向けた有用な知見を提供するものである.

\section{今後の課題}

本研究により,携帯型 CO$_2$ 測定デバイスの有効性および実環境における測定可能性が示された一方で,実用化および応用展開を見据えた際には,いくつかの課題が残されている.本節では,本研究を通じて明らかになった今後の課題について述べる.

\subsection{稼働時間の向上}

本研究では,通信方式として Wi-Fi および LTE を用いた測定機器を試作し,それぞれの省電力性能について評価を行った.その結果,テザリングや Wi-Fi 環境を利用した ESP32-C6 単体の測定機器においては,周期的な測定および待機動作を採用することで,十分な省電力動作が可能であることが確認された.一方で,LTE 通信を用いた測定機器では,通信時の消費電力が大きく,バッテリ駆動による連続稼働時間が約 16 時間程度にとどまった.

携帯型デバイスとして日常生活の中で継続的に利用することを想定した場合,1 日以上の連続稼働が可能であることが望ましい.そのため,今後は LTE 通信時の消費電力削減を目的として,通信頻度のさらなる最適化や,送信データ量の削減,通信モジュールのスリープ制御の高度化などを検討する必要がある.また,バッテリ容量の見直しや,高効率な電源回路の導入も,稼働時間向上に向けた有効な手段であると考えられる.

\subsection{測定機器のケース作成}

本研究では,測定機器の小型化を重視した設計を行ったものの,現状の試作機ではセンサ部や電子部品が外部に露出しており,持ち運び時や日常使用時における破損の危険性が残されている.また,外部からの衝撃や埃,湿気などの影響を受けやすい構造であることも課題として挙げられる.

一方で,CO$_2$ センサは周囲空気を直接取り込む必要があるため,完全に密閉されたケースを用いることは測定結果に影響を及ぼす可能性がある.そのため,今後は,通気性を確保しつつ,センサおよび回路を保護できるケース構造の検討が必要である.例えば,通気孔の配置やフィルタ材の使用,3D プリンタを用いた専用ケースの設計などが考えられる.これにより,携帯性および耐久性を向上させ,実用的な測定機器としての完成度を高めることが期待される.

\subsection{授業姿勢の判定への応用}

本研究で試作した携帯型 CO$_2$ 測定デバイスは,利用者の周囲環境における CO$_2$ 濃度を直接測定できるという特徴を有している.この特性を応用することで,教室環境における換気状態の把握だけでなく,授業中の姿勢や行動状態の推定への応用が考えられる.

例えば,着席状態で前傾姿勢をとった場合や,顔が机に近づいた場合には,呼気の影響によりセンサ近傍の CO$_2$ 濃度が一時的に上昇する可能性がある.このような CO$_2$ 濃度の変動パターンを,加速度センサや姿勢センサなどの他のセンサ情報と組み合わせることで,授業中の姿勢や集中状態を推定する手法へと発展させることが考えられる.今後は,複数センサを組み合わせたデータ取得および解析手法の検討が課題である.

\subsection{長期間運用における信頼性評価}

本研究では,短期間の測定を中心に評価を行ったが,実用化を見据えた場合,長期間にわたる連続運用時の信頼性評価が重要となる.特に,センサの経時的なドリフトや,環境条件の変化による測定精度への影響については,十分な検証が必要である.

今後は,数日から数週間にわたる連続測定を行い,測定値の安定性や再現性を評価するとともに,必要に応じて補正手法の導入を検討することが求められる.



