%序論


\section{研究背景}
2019年末に中国武漢市で報告された新型コロナウイルス感染症(COVID-19)は,瞬く間に世界中へと拡大し,2020年以降は日常生活・教育・産業活動に大きな影響を与えた.COVID-19 は飛沫感染に加え,エアロゾルを介した空気感染の可能性が指摘されており,換気の不足した室内環境において感染リスクが著しく高まることが明らかになった.厚生労働省は,密閉・密集・密接のいわゆる「三つの密」を避ける行動を促し,特に密閉空間における換気の重要性が社会全体で再認識される契機となった.

このような背景から,CO$_2$濃度を用いて換気状態を把握する手法に注目が集まった.CO$_2$は人体の呼気に大量に含まれるため,室内における滞在人数や密度を間接的に示す指標として利用でき,換気不足を判断するための有効な環境指標である.実際,コロナ禍では学校・飲食店・公共施設など多くの場面でCO$_2$センサの設置が進み,室内の換気状態を可視化する取り組みが広がった.

一方で,2023年以降は社会行動が徐々に平常化したことにより,季節性インフルエンザが全国的に大流行する状況が続いている.興味深いことに,コロナ禍の2020〜2021年にはインフルエンザの報告件数が歴史的に低水準となったことが知られている.この現象は,マスク着用や行動制限に加え,換気の徹底が呼吸器感染症全体の抑制に大きく寄与していた可能性を示唆している.

このように,COVID-19 を契機として「空気環境の重要性」は社会的に大きな注目を集めたが,感染症対策としての換気の有効性は今後も維持すべき課題であり,特に教育機関・オフィス・飲食店など多くの人が集まる空間では継続的な空気質管理が求められる.また,個人レベルでも自宅・車内・店舗・移動先など多様な環境で空気質を把握できる仕組みが求められており,より柔軟に利用できるCO$_2$測定機器の必要性が高まっている.



\section{CO$_2$濃度と空気環境}
大気中の主要成分のうち,CO$_2$はわずか 0.04\% 程度と非常に低い割合しか占めていない.しかし室内においては,人の呼気により CO$_2$濃度が急激に上昇することがあるため,換気状態を評価する代表的な指標として利用されている.CO$_2$濃度は ppm(parts per million)で表され,屋外では一般に 415〜450 ppm 程度でほぼ一定である.

厚生労働省の「建築物環境衛生管理基準」では,空気調和設備を備える居室において CO$_2$濃度を 1,000 ppm 以下に維持することが求められる.また文部科学省の「学校環境衛生基準」では,教室内の換気状態を判断するために 1,500 ppm 以下が望ましいとされている.

CO$_2$濃度が上昇すると,換気不良に加えて,集中力低下や頭痛,疲労感などの健康影響が報告されている.特に学習環境においては,CO$_2$濃度が 1,000 ppm を超えると児童生徒の認知機能が低下することが示されており,教育現場でも空気質管理の重要性が指摘されている.

このように,CO$_2$濃度は単なる環境指標としてだけでなく,人の健康や作業効率に直結する要素であり,感染症リスク管理や快適な居住環境の実現に欠かせない情報である.



\section{換気・温湿度と感染症リスク}
空気環境の指標としては CO$_2$濃度だけでなく,温度・湿度も重要である.先行研究では,低温環境や過度に乾燥した環境が呼吸器系疾患の罹患率を高めることが報告されており,適切な温湿度管理は感染症対策としても不可欠である.

COVID-19 だけでなく,インフルエンザウイルスなどの多くの呼吸器ウイルスは,乾燥した空気中で生存しやすく,また低温環境では免疫機能が低下し感染しやすくなることが知られている.さらに,換気不足の環境ではウイルスを含むエアロゾルが滞留し,同一空間内での集団感染リスクを高める.

このように,空気環境と感染症リスクは密接に関連しており,CO$_2$濃度と温湿度を総合的に把握することが,快適性の向上だけでなく,感染症予防・健康維持の観点からも重要である.



\section{現在の課題}
近年,COVID-19 を契機として室内の換気状況を可視化する手段として CO$_2$ センサの普及が急速に進んだ.しかし,現在広く利用されている多くの CO$_2$ 測定機器には,使用環境や対象に応じていくつかの課題が残されている.

第一に,既存の CO$_2$ センサの多くは据え置き型として設計されており,設置位置により測定結果が大きく変動するという問題がある.室内環境は空気の流れや空間の形状,人の動きに大きく影響されるため,換気の不十分な場所とそうでない場所が混在することがある.したがって,一台の据え置き型センサでは空間全体の換気状況を把握することが困難であり,複数台のセンサを導入する必要が生じる.しかし,複数台を設置するためにはコストや設置場所の確保といった制約があり,特に個人利用や小規模環境においては現実的ではない.

第二に,既存製品はサイズが大きく携帯性に乏しいため,利用者が自宅,大学,飲食店,車内など複数の環境を移動しながら空気質を測定するといった使い方には適していない.実際には,密閉空間や混雑した空間に入る前に換気状態を確認したい場面は日常的に多く存在するが,現状の測定器はそのような用途に十分に応えられていない.

第三に,精度の高い NDIR 式 CO$_2$ センサは価格が高く,一般利用者が複数の環境で使用するには導入コストが大きな負担となる.またセンサの精度が高くても,消費電力が大きくバッテリ駆動に向かない製品も多いため,携帯型デバイスとして長時間使用することが難しいという課題がある.

さらに,COVID-19 の流行以降,換気の重要性は社会的に広く認識されるようになった一方,2023 年以降の社会活動の再開に伴ってインフルエンザの大規模な流行が再び発生している.この状況は,感染症対策としての空気環境管理が依然として必要であることを示しているが,現状では個々人が自らの生活環境における換気状態を能動的に把握し,適切な判断を行うための手軽なツールが不足している.

以上のように,据え置き型 CO$_2$ センサでは測定場所が限定されること,小型で携帯性に優れた測定機器が不足していること,そして個人が移動先の空気環境を評価する手段が十分に整備されていないことなど,現行の環境計測機器には依然として多くの課題が存在する.

\section{本研究の目的}
本研究の目的は,従来の据え置き型 CO$_2$ センサの課題を踏まえ,携帯可能でありながら高精度な測定が可能な小型 CO$_2$ 測定デバイスを試作し,日常生活における多様な環境での空気質把握を可能にすることである.

具体的には,センサシステム全体の小型化・省電力化を図り,モバイルバッテリや内蔵電源によって長時間稼働できる CO$_2$ 測定デバイスの実現を目指す.これにより,自宅や大学といった固定的な環境だけでなく,車内,カフェ,研究室,イベント会場など,利用者が移動しながら直面するさまざまな環境の換気状況を簡便に評価できるようにする.また,NDIR 方式を採用した高精度な CO$_2$ 測定を可能とし,空気質の変化をリアルタイムで取得できるように設計することで,換気の不足を素早く検知し,感染症予防や健康維持につながる判断を支援する.

本研究では,試作した小型 CO$_2$ 測定デバイスを複数の環境で実際に使用し,CO$_2$ 濃度の変動や換気状態の違いを測定・評価する.これにより,デバイスの実用性や測定精度,省電力性能を検証し,持ち運び型空気質モニタとしての有効性を明らかにすることを目指す.さらに,測定結果を通じて,空気質の改善や適切な換気行動を促すための新たな知見を得ることも期待される.

以上の取り組みにより,個人が日常生活の中で空気環境に対する意識を高め,安全で快適な生活空間を維持するための一助となる測定デバイスの開発を目指す.