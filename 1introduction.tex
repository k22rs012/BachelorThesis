%序論


\section{研究背景}

2019年末に中国武漢市で報告された新型コロナウイルス感染症(COVID-19)は,短期間のうちに世界的な流行へと発展し,2020年以降の日常生活,教育,産業活動に深刻な影響を及ぼした.各国において外出制限や行動制限が実施され,人々の生活様式は大きく変化した.COVID-19 は飛沫感染に加え,微小な粒子であるエアロゾルを介した空気感染の可能性が指摘されており,特に換気が不十分な室内環境では感染リスクが著しく高まることが明らかになっている.この状況を受け,厚生労働省は密閉・密集・密接のいわゆる「三つの密」を回避する行動を国民に呼びかけた.なかでも密閉空間における換気の重要性は,学校や職場,公共施設などあらゆる場面で強調され,空気環境への関心が社会全体で急速に高まる契機となった.

このような背景のもと,室内の換気状態を把握するための定量的な指標として CO$_2$ 濃度に注目が集まった.CO$_2$ は人体の呼気に多く含まれるため,室内における滞在人数や人の密度を間接的に反映するものとして利用できる.換気が不十分な環境では CO$_2$ 濃度が上昇しやすく,その変化を観測することで換気状態を把握することが可能である.このため,CO$_2$ 濃度の上昇は換気不足を示す重要なサインとなり,空気環境評価において有効な指標として位置付けられている.実際にコロナ禍では,学校,飲食店,公共施設など多くの場所で CO$_2$ センサの設置が進められ,数値によって換気状態を可視化する取り組みが広く実施された.これにより,CO$_2$ 濃度は感染症対策を支援する実用的な環境指標として社会に定着したといえる.

換気の重要性はCOVID-19 に限らず,季節性インフルエンザをはじめとする呼吸器感染症においても指摘されている。インフルエンザウイルスは空気が乾燥した環境において生存しやすいことが知られており,室内環境の管理が感染拡大に影響を及ぼす可能性がある.このことから,マスク着用や人流の抑制に加え,換気の徹底によって室内の空気環境を適切に維持することが,インフルエンザの感染拡大抑制に寄与すると考えられる.すなわち,換気はインフルエンザに対しても有効な感染対策の一つである.

以上のように,COVID-19 を契機として空気環境の重要性は社会的に広く認識されるようになったが,換気の有効性は一時的な感染症対策にとどめるべきではない.特に,教育機関,オフィス,飲食店など,多くの人が集まる空間では,平常時においても継続的な空気質管理が求められる.さらに,個人レベルにおいても,自宅,車内,店舗,移動先など多様な環境で空気質を把握できる仕組みが重要である.しかし,既存の CO$_2$ センサの多くは据え置き型であり,設置場所が限定されるという課題がある.このような背景から,柔軟に持ち運びが可能で,日常生活の中でさまざまな環境における空気質を把握できる CO$_2$ 測定機器の必要性が高まっている.






\section{CO$_2$濃度と空気環境}

CO$_2$濃度は,室内の空気環境を評価するための代表的な指標の一つである.大気中の主要成分のうち,CO$_2$は約 0.04\% と非常に低い割合しか占めていないが,室内空間では人の呼気によって短時間で濃度が上昇する特徴を持つ.そのため,CO$_2$濃度の変化は,室内における人の滞在状況や換気の十分さを間接的に反映することから,空気質評価において重要な指標として広く利用されている.

CO$_2$濃度は ppm(parts per million)を単位として表され,屋外環境では一般に 415〜450 ppm 程度でほぼ一定の値を示す.一方,換気が不十分な室内環境では,人の呼吸活動により CO$_2$濃度が急激に上昇することが知られている.このため,屋外濃度と室内濃度の差を観測することで,換気の十分さを定量的に評価することが可能である.特に,多人数が長時間滞在する空間では,CO$_2$濃度の継続的な監視が重要となる.

CO$_2$濃度に関しては,国の指針や基準においても具体的な数値が示されている.厚生労働省の「建築物環境衛生管理基準」では,空気調和設備を備える居室空間において,CO$_2$濃度を 1,000 ppm 以下に維持することが求められている.また,文部科学省の「学校環境衛生基準」では,教室内の換気状態を判断する目安として 1,500 ppm 以下が望ましいとされている.これらの基準は,CO$_2$濃度が室内空気環境の良否を判断する実用的な指標であることを示している.

さらに,CO$_2$濃度の上昇は換気不良を示すだけでなく,人の健康や作業・学習効率にも影響を及ぼすことが報告されている.具体的には,CO$_2$濃度の上昇に伴い,集中力の低下,頭痛,眠気,疲労感などの症状が現れる可能性が指摘されている.特に学習環境においては,教室内のCO$_2$濃度が 1,000 ppm を超えると児童生徒の認知機能や学習効率が低下することが示されており,教育現場における空気質管理の重要性が強調されている.

以上のように,CO$_2$濃度は換気状態を把握するための環境指標であると同時に,人の健康や快適性,作業・学習効率に直結する重要な要素である.そのため,感染症リスクの低減や快適な居住・活動環境を実現する上で,CO$_2$濃度を継続的に把握し,適切な換気を行うことが不可欠である.

\begin{table}[htbp]
\centering
\caption{場所別にみた CO$_2$ 濃度の目安}
\label{tab:co2_place_guideline}
\begin{tabular}{|c|c|}
\hline
\textbf{測定場所} & \textbf{CO$_2$ 濃度の目安} \\ \hline \hline
屋外環境 & 約 415~450 ppm 程度 \\ \hline
居住空間(住宅・オフィス) & 1,000 ppm 以下が望ましい \\ \hline
教育施設(教室) & 1,500 ppm 以下が推奨される \\ \hline
\end{tabular}
\end{table}

\begin{table}[htbp]
\centering
\caption{CO$_2$ 濃度と人体・環境への影響の整理}
\label{tab:co2_health_effect}
\small
\begin{tabular}{|c|c|c|c|}
\hline
\textbf{CO$_2$ 濃度} & \textbf{生理的影響} & \textbf{認知・行動への影響} & \textbf{評価・位置づけ} \\ \hline \hline
500 ppm 程度 & 特に影響なし & 快適な環境 & 屋外環境に近い状態 \\ \hline
1,000 ppm 超 & 軽度の眠気・集中力低下の可能性 & 判断力・注意力の低下 & 室内環境の管理目安 \\ \hline
1,500 ppm 超 & 不快感の増加 & 学習効率の低下 & 換気改善が望ましい水準 \\ \hline
5,000 ppm 超 & 呼吸数の増加 & 作業効率の低下 & 労働環境の管理基準 \\ \hline
10,000 ppm 超 & 呼吸困難・頭痛 & 正常な作業が困難 & 危険レベル \\ \hline
50,000 ppm 超 & めまい・意識障害 & 行動不能 & 生命に危険 \\ \hline
\end{tabular}
\end{table}

\FloatBarrier


\section{換気・温湿度と感染症リスク}
空気環境の指標としては CO$_2$濃度だけでなく,温度・湿度も重要である.先行研究では,低温環境や過度に乾燥した環境が呼吸器系疾患の罹患率を高めることが報告されており,適切な温湿度管理は感染症対策としても不可欠である.

COVID-19 をはじめ,インフルエンザウイルスなどの多くの呼吸器ウイルスは神吉条件に影響されやすく,低温環境では免疫機能が低下し感染しやすくなることが知られている.さらに,換気不足の環境ではウイルスを含むエアロゾルが滞留し,同一空間内での集団感染リスクを高める.

このように,空気環境と感染症リスクは密接に関連しており,CO$_2$濃度と温湿度を総合的に把握することが,快適性の向上だけでなく,感染症予防・健康維持の観点からも重要である.




\section{現在の課題}
近年,COVID-19 を契機に室内の換気状況を可視化する手段として CO$_2$ センサが急速に普及した.しかし,現在広く利用されている多くの CO$_2$ 測定機器には,使用環境や対象に応じていくつかの課題が残されている.

第一に,既存の CO$_2$ センサの多くは据え置き型として設計されており,設置位置により測定結果が大きく変動するという問題がある.室内環境は空気の流れや空間の形状,人の動きに大きく影響されるため,換気の十分な場所とそうでない場所が混在することがある.したがって,一台の据え置き型センサでは空間全体の換気状況を把握することが困難であり,複数台のセンサを導入する必要が生じる.しかし,複数台を設置するためにはコストや設置場所の確保といった制約があり,特に個人利用や小規模環境においては現実的ではない.

第二に,既存製品はサイズが大きく携帯性に乏しいため,自宅,大学,飲食店,車内など複数の環境を移動しながら空気質を測定したい利用者のニーズに応えられず,実際には密閉空間や混雑した空間に入る前に換気状態を確認したい場面が日常的に多く存在するにもかかわらず,現状の測定器では十分に対応できない.

第三に,精度の高いとされている NDIR(Non-Dispersive Infrared) 式 CO$_2$ センサは信頼性に優れる一方で価格が高く,一般利用者が複数の環境で使用するには導入コストが大きな負担となる.またセンサの精度が高くても,消費電力が大きくバッテリ駆動に向かない製品も多いため,携帯型デバイスとして長時間使用することが難しいという課題がある.

これらの課題は,個人が自らの生活環境における換気状態を能動的に把握し,適切な判断を行うための手軽なツールが不足していることを示しており,感染症対策において,室内空気環境を適切に管理することの重要性を改めて示している.

以上のように,据え置き型 CO$_2$ センサでは測定場所が限定されること,小型で携帯性に優れた測定機器が不足していること,そして個人が移動先の空気環境を評価する手段が十分に整備されていないことなど,現状の環境計測機器には依然として多くの課題が存在する.

\section{先行研究}

これまでに,室内環境の換気状態を把握することを目的として,CO$_2$ センサを用いたさまざまな研究が行われてきた.特に,COVID-19 の流行以降は,学校や公共施設などの室内空間における換気状況を定量的に評価する手法として,据え置き型 CO$_2$ センサを用いた測定システムに関する研究が多く報告されている.これらの研究では,複数台の CO$_2$ センサを室内に設置し,空間内の CO$_2$ 濃度分布や時間変化を可視化することで,換気状態を評価する手法が提案されている.

例えば,室内に複数の CO$_2$ センサを配置し,高さ方向や位置ごとの CO$_2$ 濃度差を測定することで,換気の偏りや滞留しやすい領域を明らかにする研究が報告されている.これにより,空調設備や換気経路の改善に向けた有用な知見が得られており,室内環境の評価手法として高い有効性を示している.また,測定結果をサーバに集約し,PC やスマートフォン上で時系列グラフとして可視化するシステムも提案されており,換気状態を視覚的に把握できる点が評価されている.

しかし,これらの先行研究で用いられている測定システムは多くが据え置き型であり、前述の課題を抱えている.さらに,測定対象は空間全体に偏る傾向があり,個人が実際に滞在・移動する位置での空気環境を直接評価することは困難である.このことから,先行研究は室内環境全体の評価という観点では有用である一方,個人の生活環境における簡便な空気質把握手段としては十分とはいえない.

以上のことから,従来の据え置き型 CO$_2$ センサを用いた研究は,室内空気環境の評価において重要な役割を果たしてきたものの,携帯性や個人単位での環境把握という点においては課題が残されている.これらの課題を踏まえ,利用者が移動しながらさまざまな環境における空気質を把握できる,小型で携帯可能な CO$_2$ 測定機器の開発が求められている.



\section{本研究の目的}
本研究の目的は,従来の据え置き型 CO$_2$ センサの課題を踏まえ,携帯可能でありながら高精度な測定が可能な小型 CO$_2$ 測定デバイスを試作し,日常生活における多様な環境での空気質把握を可能にすることである.

具体的には,センサシステム全体の小型化・省電力化を図り,モバイルバッテリや内蔵電源によって長時間稼働できる CO$_2$ 測定デバイスの実現を目指す.これにより,自宅や大学といった固定的な環境だけでなく,車内,カフェ,研究室,イベント会場など,利用者が移動しながら直面するさまざまな環境の換気状況を簡便に評価できるようにする.また,NDIR 方式を採用した高精度な CO$_2$ 測定を可能とし,空気質の変化をリアルタイムで取得できるように設計することで,換気の不足を素早く検知し,感染症予防や健康維持につながる判断を支援する.

本研究では,試作した小型 CO$_2$ 測定デバイスを複数の環境で実際に使用し,CO$_2$ 濃度の変動や換気状態の違いを測定・評価する.これにより,デバイスの実用性や測定精度,省電力性能を検証し,持ち運び型空気質モニタとしての有効性を明らかにすることを目指す.さらに,測定結果を通じて,空気質の改善や適切な換気行動を促すための新たな知見を得ることも期待される.

以上の取り組みにより,個人が日常生活の中で空気環境に対する意識を高め,安全で快適な生活空間を維持するための一助となる測定デバイスの開発を目指す.