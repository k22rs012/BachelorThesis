% 評価方法


% ↓比較対象たち 
% https://gillinstruments.com/compare-2-axis-anemometers/windobserver-2axis/
% https://gillinstruments.com/wp-content/uploads/2022/08/1390-0036-Iss-10-WindObserver65.pdf
% https://www.fcmarine.co.uk/product/windobserver-65-ultrasonic-anemometer/?srsltid=AfmBOoqQc0qhhxDZMxvWMz4hp1mfAXZEo-3P--7X_9c0VlrvbNU4rxmm&utm_source=chatgpt.com

% https://www.weather.jp/products/weather-transmitter/cvs-wxt530/#%E8%A9%B3%E7%B4%B0%E4%BB%95%E6%A7%98

% https://www.amazon.com/dp/B087V6N5WZ?ref=cm_sw_r_cp_ud_dp_JGM2FGRT6FMAYPWHD2MA&ref_=cm_sw_r_cp_ud_dp_JGM2FGRT6FMAYPWHD2MA&social_share=cm_sw_r_cp_ud_dp_JGM2FGRT6FMAYPWHD2MA&newOGT=1&th=1

% https://www.snk.co.jp/service/technology/search/?pdid=400

% https://satosokuteiki.com/item/detail/5126

% https://www.tekhne.co.jp/product_catalog/ee65-ee66.pdf



% 風速の精度と稼働時間

\section{概要}

本章では,本研究で試作した小型 CO$_2$ 測定デバイスを用い,従来の据え置き型 CO$_2$ センサと比較して,同一環境において同様の CO$_2$ 濃度変化が得られるかを評価する。また,測定デバイスを携帯した状態で,様々な環境において問題なく CO$_2$ 濃度を測定できるかを検証する。はじめに,福岡県赤村に設置されているドームハウス内において,従来の据え置き型 CO$_2$ センサを 35 台設置し,それぞれの近傍に本研究で試作した小型 CO$_2$ 測定デバイスを配置した。この環境において両者の測定値を比較し,CO$_2$ 濃度の時間変化が同様の傾向を示すかを確認した。次に,携帯型デバイスとしての有効性を評価するため,測定デバイスを首の前,首の後ろ,腰の前,腰の後ろの計 4 箇所に装着し,赤村周辺の山道を登る測定を行った。この測定では,装着位置や移動によってCO$_2$ 濃度に異常な変動が生じないかを確認した。
以下では,各測定環境および測定方法について詳述する。

\section{測定環境}
  \subsection{赤村ドームハウス}
  赤村のドームハウスは,複数人が滞在可能な閉鎖空間であり,換気状態や人の滞在による CO$_2$ 濃度変化を評価するのに適した環境である。本測定では,ドームハウス内に従来の据え置き型 CO$_2$ センサを 35 台設置し,それぞれの近傍に小型 CO$_2$ 測定デバイスを配置した。
  
  \subsection{山道における携帯測定環境}

\section{測定方法}
  \subsection{据え置き型センサとの比較測定}

据え置き型 CO$_2$ 測定器との比較測定では,本研究で試作した小型 CO$_2$ 測定デバイスを据え置き型測定器の直近に設置し,両者の測定値を同一環境下で取得した。

具体的には,小型 CO$_2$ 測定デバイスを据え置き型測定器の隣に配置して測定を行い,測定終了後に設置位置を一段上へ移動させる操作を繰り返した。この操作を EXAKA1 から EXAKA35 までの全ての据え置き型測定器に対して順に実施し,合計 35 箇所における比較測定を行った。測定時の小型 CO$_2$ 測定デバイスと据え置き型測定器の配置関係を図\ref{fig:portable_vs_fixed} に示す。本図に示すように,小型測定デバイスは各据え置き型測定器の近傍に設置されており,同一高さ・同一空間におけるCO$_2$ 濃度を測定できるよう配慮した。
  \subsection{携帯時の測定方法}

\section{測定条件}
  \subsection{測定間隔}
  \subsection{装着位置}
  \subsection{測定時間}

\section{評価方法}


