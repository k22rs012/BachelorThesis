% 評価方法

\section{概要}

本章では,本研究で試作した小型 CO$_2$ 測定デバイスを用い,従来の据え置き型 CO$_2$ センサと比較して,同一環境において同様の CO$_2$ 濃度変化が得られるかを評価する。また,測定デバイスを携帯した状態で,様々な環境において問題なく CO$_2$ 濃度を測定できるかを検証する。はじめに,福岡県赤村に設置されているドームハウス内において,従来の据え置き型 CO$_2$ センサを 35 台設置し,それぞれの近傍に本研究で試作した小型 CO$_2$ 測定デバイスを配置した。この環境において両者の測定値を比較し,CO$_2$ 濃度の時間変化が同様の傾向を示すかを確認した。次に,携帯型デバイスとしての有効性を評価するため,測定デバイスを首の前,首の後ろ,腰の前,腰の後ろの計 4 箇所に装着し,赤村周辺の山道を登る測定を行った。この測定では,装着位置や移動によってCO$_2$ 濃度に異常な変動が生じないかを確認した。
以下では,各測定環境および測定方法について詳述する。

\section{測定環境}
  \subsection{赤村ドームハウス}
  赤村のドームハウスは,複数人が滞在可能な閉鎖空間であり,換気状態や人の滞在による CO$_2$ 濃度変化を評価するのに適した環境である。本測定では,ドームハウス内に従来の据え置き型 CO$_2$ センサを 35 台設置し,それぞれの近傍に小型 CO$_2$ 測定デバイスを配置した。
  
  \subsection{山道における携帯測定環境}
山道における携帯測定は,福岡県赤村に位置する岩石山において実施した。岩石山は標高約 454 m の山であり,登山道が整備されたハイキングコースを有している。本測定では,屋外の移動環境において小型 CO$_2$ 測定デバイスを携帯した状態での測定が可能であるかを評価することを目的とした。測定は登山中に実施し,小型 CO$_2$ 測定デバイスを首の前,首の後ろ,腰の前,腰の後ろの計 4 箇所に装着した。これにより,装着位置や身体の動きによってCO$_2$ 濃度の測定値に異常な変動が生じないかを確認した。

\section{測定方法}
 \subsection{据え置き型センサとの比較測定}

本測定では,据え置き型 CO$_2$ 測定器を複数台設置することで得られる室内の CO$_2$ 濃度分布を基準とし,携帯型の小型 CO$_2$ 測定デバイスが換気状態の把握という目的に対してどの程度有効であるかを評価することを目的とした。そのため,小型測定デバイスを据え置き型測定器の設置位置ごとに順次配置し,同一高さ・同一環境における測定値の比較を行った。

比較測定は,福岡県赤村のドームハウス内に設置された据え置き型 CO$_2$ 測定器を対象として実施した。本測定では,据え置き型 CO$_2$ 測定器を合計 35 台使用し,これらは螺旋階段および 2 階部分に設置された棚を含め,床面付近から天井付近までの高さ方向に配置されている。各測定器には EXAKA1 から EXAKA35 の識別子を付与し,EXAKA1 を最下部,EXAKA35 を最上部とした。

比較測定の手順として,小型 CO$_2$ 測定デバイスを各据え置き型測定器の直近に配置して測定を行い,測定終了後に設置位置を一段上へ移動させる操作を繰り返した。この操作を EXAKA1 から EXAKA35 までの全ての据え置き型測定器に対して順に実施し,合計 35 箇所における比較測定を行った。測定時の小型 CO$_2$ 測定デバイスと据え置き型 CO$_2$ 測定器の配置関係を図\ref{fig:portable_vs_fixed} に示す。本図に示すように,小型測定デバイスは各据え置き型測定器の近傍に設置されており,同一高さ・同一空間におけるCO$_2$ 濃度を測定できるよう配慮した。


  \subsection{携帯時の測定方法}
 本測定では 、図〜に示すように

\section{測定条件}
  \subsection{測定間隔}
  \subsection{装着位置}
  \subsection{測定時間}

\section{評価方法}


