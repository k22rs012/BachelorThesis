% 評価方法

\section{概要}

本章では,本研究で試作した小型 CO$_2$ 測定デバイスを用い,従来の据え置き型 CO$_2$ センサと比較して,同一環境において同様の CO$_2$ 濃度変化が得られるかを評価する。また,測定デバイスを携帯した状態で,様々な環境において問題なく CO$_2$ 濃度を測定できるかを検証する。はじめに,福岡県赤村に設置されているドームハウス内において,従来の据え置き型 CO$_2$ センサを 35 台設置し,それぞれの近傍に本研究で試作した小型 CO$_2$ 測定デバイスを配置した。この環境において両者の測定値を比較し,CO$_2$ 濃度の時間変化が同様の傾向を示すかを確認した。次に,携帯型デバイスとしての有効性を評価するため,測定デバイスを首の前,首の後ろ,腰の前,腰の後ろの計 4 箇所に装着し,赤村周辺の山道を登る測定を行った。この測定では,装着位置や移動によってCO$_2$ 濃度に異常な変動が生じないかを確認した。
以下では,各測定環境および測定方法について詳述する。

\section{測定環境}
  \subsection{赤村ドームハウス}
  赤村のドームハウスは,複数人が滞在可能な閉鎖空間であり,換気状態や人の滞在による CO$_2$ 濃度変化を評価するのに適した環境である。本測定では,ドームハウス内に従来の据え置き型 CO$_2$ センサを 35 台設置し,それぞれの近傍に小型 CO$_2$ 測定デバイスを配置した。
  
  \subsection{山道における携帯測定環境}
山道における携帯測定は,福岡県赤村に位置する岩石山において実施した。岩石山は標高約 454 m の山であり,登山道が整備されたハイキングコースを有している。本測定では,屋外の移動環境において小型 CO$_2$ 測定デバイスを携帯した状態での測定が可能であるかを評価することを目的とした。測定は登山中に実施し,小型 CO$_2$ 測定デバイスを首の前,首の後ろ,腰の前,腰の後ろの計 4 箇所に装着した。これにより,装着位置や身体の動きによってCO$_2$ 濃度の測定値に異常な変動が生じないかを確認した。

\section{測定方法}
 \subsection{据え置き型センサとの比較測定}

本測定では,据え置き型 CO$_2$ 測定器を複数台設置することで得られる室内の CO$_2$ 濃度分布を基準とし,携帯型の小型 CO$_2$ 測定デバイスが換気状態の把握という目的に対してどの程度有効であるかを評価することを目的とした。そのため,小型測定デバイスを据え置き型測定器の設置位置ごとに順次配置し,同一高さ・同一環境における測定値の比較を行った。

比較測定は,福岡県赤村のドームハウス内に設置された据え置き型 CO$_2$ 測定器を対象として実施した。本測定では,据え置き型 CO$_2$ 測定器を合計 35 台使用し,これらは螺旋階段および 2 階部分に設置された棚を含め,床面付近から天井付近までの高さ方向に配置されている。各測定器には EXAKA1 から EXAKA35 の識別子を付与し,EXAKA1 を最下部,EXAKA35 を最上部とした。

比較測定の手順として,小型 CO$_2$ 測定デバイスを各据え置き型測定器の直近に配置して測定を行い,測定終了後に設置位置を一段上へ移動させる操作を繰り返した。この操作を EXAKA1 から EXAKA35 までの全ての据え置き型測定器に対して順に実施し,合計 35 箇所における比較測定を行った。測定時の小型 CO$_2$ 測定デバイスと据え置き型 CO$_2$ 測定器の配置関係を図に示す。本図に示すように,小型測定デバイスは各据え置き型測定器の近傍に設置されており,同一高さ・同一空間におけるCO$_2$ 濃度を測定できるよう配慮した。


  \subsection{携帯時の測定方法}

携帯時の測定では,小型 CO$_2$ 測定デバイスを利用者の身体に装着し,登山中に連続して CO$_2$ 濃度を測定した。装着位置は,首の前,首の後ろ,腰の前,腰の後ろの計 4 箇所とした。測定中は通常の歩行動作を行い,移動や姿勢変化によって測定値に大きな変動や異常値が発生しないかを確認した。

\section{測定条件}
\subsection{測定間隔}

山道における測定では,スマートフォンのテザリング接続が一定時間通信を行わない場合に切断される特性を考慮し,測定間隔を 30 秒とした。これにより,通信が途切れることなく安定して測定データを取得できるようにした。ドームハウスでの測定では,各測定位置において小型 CO$_2$ 測定デバイスを据え置き型測定器の近傍に設置して測定を行い,測定終了後は速やかに次の測定位置へ移動した。

\subsection{装着位置}
装着位置は,日常的な携帯を想定し,首および腰の前後の 4 箇所とした。

\subsection{測定時間}

ドームハウスにおける測定では,最下部に設置された測定器から最上部に設置された測定器までの全測定位置において比較測定を行い,測定全体に要した時間は約 30 分間であった。山道における測定では,登山開始から登頂および下山完了までの約 2 時間にわたり測定を行った。

\section{評価方法}
  \subsection{CO$_2$ 濃度測定精度の評価方法}

CO$_2$ 濃度測定精度の評価は,据え置き型 CO$_2$ 測定器を基準として,本研究で試作した小型 CO$_2$ 測定デバイスの測定結果を比較することで行った。評価では,測定値の絶対値の一致ではなく,時間変化の傾向が一致しているかに着目した。具体的には,同一位置・同一環境において取得した据え置き型測定器および小型測定デバイスのCO$_2$ 濃度の時間変化を比較し,両者が同様の増減傾向を示す場合に,測定精度が十分であると判断した。

  \subsection{携帯利用時の測定安定性の評価方法}
  \subsection{通信機能の評価方法}
  \subsection{省電力性能の評価方法}

省電力性能の評価は,ドームハウスおよび山道での測定とは別に,教室内において実施した。本評価では,測定機器を通常の測定動作状態で動作させ,一定時間あたりの消費電力を測定することで,長時間利用の可能性を評価した。

具体的には,測定機器を 10 分間動作させ,その間の消費電力を計測した。得られた消費電力量を基に,使用しているリチウムポリマーバッテリの総電力量と比較することで,連続動作時における動作可能時間を推定した。この方法により,屋外環境を含む様々な場所で長時間利用するという設計仕様に対して,省電力設計が有効であるかを評価した。

