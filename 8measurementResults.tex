% 評価結果


% 消費電力:USBテスタ(AVHzYct-3)を使用して測定し、消費電力を確認。
% 安定性:波形データから安定性を評価し、機器の安定した計測・データ送信の有無を確認。
% 稼働時間:Webシステムにデータが表示されなくなるまでの時間を測定。
% 精度:市販のTesto 405iと比較して、開発したセンサの精度を確認。
% 寸法:各機器のサイズ比較。
% 価格:各機器のコストパフォーマンスを評価。
% 他社サービスとの比較:他社の製品やサービス(Testo 405i、Wi-Musu、Gill WindObserver 65など)との比較。




\section{概要}

本章では,第5章で述べた測定方式および試作した小型 CO$_2$ 測定デバイスを用いて実施した各種測定実験の結果を示し,据え置き型 CO$_2$ センサとの比較を通じて,携帯型測定デバイスの特性を評価することを目的とする。はじめに,福岡県赤村に設置されたドームハウスを対象として,高さ方向における CO$_2$ 濃度分布の測定を行う。ドームハウス内には床付近から天井付近にかけて複数の据え置き型 CO$_2$ センサを設置し,空間全体の CO$_2$ 濃度分布を把握する。また,これらの測定結果を基準として,本研究で試作した携帯型 CO$_2$ 測定デバイスによる比較測定を行い,高さ方向における CO$_2$ 濃度変化の傾向を比較する。次に,携帯型 CO$_2$ 測定デバイスを実際の生活環境で使用した際の測定結果について示す。具体的には,装着位置の違いによる測定結果の比較,登山中および電車内といった移動環境における CO$_2$ 濃度の時間変化を測定し,環境条件や混雑状況の変化が測定結果に与える影響を検討する。さらに,測定機器の省電力性能および小型化に関する評価を行う。通信方式の異なる測定機器を用いたバッテリ駆動時の稼働時間の比較や,試作機ごとの外形寸法の整理を通じて,携帯型 CO$_2$ 測定デバイスとしての実用性について検討する。
以下では,各測定環境における測定結果および評価について順に述べる。

\section{据え置き型センサとの比較結果}

\subsection{据え置き型 CO$_2$ センサによる測定結果}
据え置き型 CO$_2$ センサを用いてドームハウス内の CO$_2$ 濃度分布を測定した結果を図 \ref{fig:height_co2_comparison} に示す。据え置き型センサによる測定結果では,階段下から階段上部,さらに 2 階上部へと高さが増加するにつれて,CO$_2$ 濃度が段階的に上昇する傾向が確認された。特に,階段上部および 2 階上部では,階段下部と比較して高い CO$_2$ 濃度が観測されており,ドームハウス内部において CO$_2$ が上部に滞留しやすい空間構造を有していることが示唆される。この結果は,空気の対流や換気経路の影響により,高さ方向に CO$_2$ 濃度の偏りが生じている可能性を示している。


\subsection{小型 CO$_2$ 測定デバイスによる測定結果}
次に,本研究で試作した携帯型 CO$_2$ 測定デバイスを用いて,同様にドームハウス内の高さ方向における測定を行った。その結果,小型デバイスにおいても,据え置き型 CO$_2$ センサと同様に,高さの増加に伴って CO$_2$ 濃度が上昇する傾向が確認された。小型デバイスによる測定値は,据え置き型センサの測定値と比較して全体的に高い値を示しているが,これはキャリブレーション条件や測定環境の違いによる影響であると考えられる。一方で,階段上部および 2 階上部において CO$_2$ 濃度が高くなるという相対的な変化の傾向は,据え置き型センサの結果と概ね一致している。

\begin{table}[H]
\centering
\caption{ドームハウス内における高さ方向のCO$_2$濃度測定結果}
\label{tab:dome_co2_comparison}
\begin{tabular}{lcc}
\hline
測定位置 & 据え置き型CO$_2$濃度 [ppm] & 携帯型CO$_2$濃度 [ppm] \\
\hline
階段下部 & 701.0 & 1153.8 \\
階段中部 & 739.4 & 1184.6 \\
階段上部 & 871.0 & 1279.6 \\
2階下部 & 749.2 & 1170.4 \\
2階中下 & 824.6 & 1172.8 \\
2階中部 & 828.4 & 1233.6 \\
2階上部 & 836.6 & 1264.8 \\
\hline
\end{tabular}
\end{table}

\subsection{据え置き型センサとの比較評価}

図 \ref{fig:height_co2_comparison} は,据え置き型 CO$_2$ センサおよび携帯型 CO$_2$ 測定デバイスによる測定結果を,各測定点を 5 点ずつ平均化し,高さ方向に 7 区分して比較したものである。本図より,両者の測定結果は絶対値こそ異なるものの,高さ方向に沿った CO$_2$ 濃度の遷移の仕方が類似していることが分かる。特に,階段下部から階段上部にかけての上昇傾向,および 2 階上部における高濃度領域の検出は,両者で共通して確認された。このことから,小型 CO$_2$ 測定デバイスは,多数の据え置き型センサを用いなくても,空間内における CO$_2$ 濃度分布の特徴や,相対的に換気が不十分な領域を把握できる可能性が示された。以上の結果より,本研究で試作した携帯型 CO$_2$ 測定デバイスは,利用者が任意の位置で測定を行うことで,空間内の危険性が高い領域を簡便に察知する手段として有効であると考えられる。

\begin{figure}[H]
\centering
\includegraphics[width=0.9\linewidth]{./figures/赤村比較}
\caption{
据え置き型と携帯型のCO$_2$濃度比較
}
\label{fig:height_co2_comparison}
\end{figure}


\section{携帯時の測定結果}
  \subsection{装着位置による測定結果}
  
  
  
   \subsection{電車内の CO$_2$ 濃度変化}
図Xに、電車内において異なる位置で測定した CO$_2$ 濃度の時間変化を示す。オレンジの線はドア付近(CS2)、青の線はドアから離れた位置(CS1)で測定した結果である。

測定区間全体を通して、両測定点のCO$_2$濃度は同様の増減傾向を示したが、濃度の絶対値には明確な差が確認された。特に、複数の時間帯においてドアから離れた位置(CS1)の  CO$_2$濃度が、ドア付近(CS2)よりも高くなる傾向が見られた。例えば、20:55 頃には CS1 で約 1300 ppm を超える値が観測された一方、同時刻の CS2 ではそれより低い値を示している。また、各駅での停車前後に着目すると、ドア開放直後に CO$_2$濃度が低下する傾向が確認された。これは、ドア開放に伴う外気の流入により換気が促進されたためと考えられる。この低下はドア付近の測定点(CS2)でより顕著であり、ドアから離れた位置では低下量が小さい、もしくは遅れて現れる場合があった。さらに、博多駅など乗降客が多い駅では、停車後に CO$_2$濃度が一時的に上昇する傾向も観測された。これは、乗客増加に伴う呼気由来の CO$_2$発生量の増加が、換気量を上回ったためと推測される。一方、弥生ヶ丘駅付近のように車内の乗客数が少ない区間では、全体的に CO$_2$濃度が低下し、両測定点の差も縮小する傾向が見られた。これらの結果から、同一車両内であっても測定位置によって CO$_2$濃度、すなわち換気状況に差が生じることが明らかとなった。特に、ドアから離れた位置では換気の影響を受けにくく、CO$_2$濃度が高くなりやすい傾向が示された。

\begin{figure}[H]
\centering
\includegraphics[width=0.5\linewidth]{./figures/densyanai}
\caption{
電車内のco2濃度変化
}
\label{fig:densyanai}
\end{figure}
\FloatBarrier

\section{省電力性能の評価結果}
本節では,測定機器3および測定機器4において,バッテリ駆動時の稼働時間を測定し,通信方式の違いが消費電力に与える影響について評価を行う。各測定機器には容量 25\,mAh のリチウムポリマバッテリを使用し,測定間隔は 5 分とした。測定時には CO$_2$ 濃度の取得および必要な通信処理を行い,それ以外の時間は ESP32-C6 を DeepSleep 状態へ移行させる構成とした。この条件の下で,バッテリ電圧の時間変化および連続稼働時間を測定した。LTE 通信モジュール SIM7080G を搭載した測定機器4では,0 時から 15 時までの約 15 時間にわたり連続動作することを確認した。図\ref{fig:mune} に示すように,動作開始直後は約 4.1\,V 付近であったバッテリ電圧が,時間の経過とともに徐々に低下し,動作終了時には約 3.0\,V まで低下した。この間,設定した測定間隔で測定および通信処理が正常に実行されており,LTE 通信を含む構成においてもDeepSleep を用いた周期動作が実現できていることが確認できた。


\begin{figure}[H]
\centering
\includegraphics[width=0.5\linewidth]{./figures/voltage}
\caption{
測定機器4のバッテリ持続時間
}
\label{fig:densyanai}
\end{figure}
\FloatBarrier


一方で,LTE 通信を使用しない測定機器3について評価を行った。測定機器3では,Wi-Fi を用いて測定データを送信する構成であり,LTE 通信モジュールを搭載していない点が特徴である。そのため,セルラー通信に比べて通信時の消費電力を抑えた動作が可能である。その結果,同一条件(測定間隔 5 分,DeepSleep 使用,バッテリ容量 25\,mAh)において,約 56 時間の連続稼働を確認した。この結果から,Wi-Fi 通信を用いた構成においても,DeepSleep を併用することで小容量バッテリによる長時間動作が可能であることが示された。これらの結果から,通信方式の違いが測定機器の消費電力および稼働時間に大きく影響することが分かる。LTE 通信を用いた測定機器4は,屋内外を問わず単体で通信可能であるという利点を持つ一方,消費電力が増加し,稼働時間は約 15 時間となった。一方で,LTE 通信を使用しない測定機器3では,通信機能が制限されるものの,約 56 時間の連続稼働が可能であり,省電力性に優れた構成であることが確認された。

\begin{figure}[H]
\centering
\includegraphics[width=0.5\linewidth]{./figures/voltage}
\caption{
測定機器3のバッテリ持続時間
}
\label{fig:densyanai}
\end{figure}
\FloatBarrier

以上より,本研究で提案する測定機器は,利用目的に応じて通信方式を選択することで,携帯性と稼働時間のバランスを調整できることが示された。


\section{小型化に関する評価}

本節では,本研究で段階的に開発した測定機器1~4について,筐体の小型化の観点から評価を行う。携帯型 CO$_2$ 測定デバイスとして日常生活での利用を想定した場合,測定精度や通信性能に加えて,機器の大きさは携帯性や使用頻度に大きく影響する重要な要素である。そのため,本研究では各試作機の外形寸法を整理し,小型化の進展について定量的および定性的に評価を行った。

表\ref{tab:size_comparison}に,測定機器1~4の外形寸法の比較を示す。本表では,各測定機器の外形寸法を縦および横の寸法で示しており,試作機ごとの構成や改良点を併せて整理している。

\begin{table}[htbp]
\centering
\caption{測定機器1~4の外形寸法の比較}
\label{tab:size_comparison}
\begin{tabular}{|c|c|c|}
\hline
測定機器 & 外形寸法 [mm] & 備考 \\ \hline \hline
測定機器1 & 約 40 $\times$ 60 & 初期試作機 \\ \hline
測定機器2 & 約 50 $\times$ 70 & LED追加 \\ \hline
測定機器3 & 約 30 $\times$ 40 & 小型化試作機 \\ \hline
測定機器4 & 約 35 $\times$ 55 & LTE通信対応 \\ \hline
\end{tabular}
\end{table}

測定機器1は,ESP32-C6,SCD41,プッシュボタンおよびリチウムポリマバッテリを用いた最小構成の試作機であり,外形寸法は約 40\,mm $\times$ 60\,mm であった。本機器は機能確認を主目的として作成したため,部品配置の最適化や筐体サイズの削減は行っておらず,比較的余裕のある寸法となっている。

測定機器2では,測定機器1の構成を維持したまま利用者への視覚的フィードバックを目的として LED を追加した。その結果,外形寸法は約 50\,mm $\times$ 70\,mm となり,測定機器1と比較してやや大型化している。これは,機能追加を優先した設計によるものであり,試作段階における仕様上の選択である。

測定機器3では,測定機器2で確認された機能を維持しつつ,携帯性の向上を目的として筐体の小型化を行った。部品配置を見直し,ESP32-C6 の表面および裏面の両方に部品を実装する構成とすることで,外形寸法は約 30\,mm $\times$ 40\,mm まで縮小された。測定機器1と比較すると,外形寸法(縦×横)から算出した外形面積は,約 2400\,mm$^2$ から約 1200\,mm$^2$ へと減少しており,約 50\% の削減が達成されている。この結果から,携帯型測定機器としての実用性が大きく向上していることが分かる。

さらに,測定機器4では,測定機器3で達成した小型化を意識しつつ,LTE 通信モジュールである SIM7080G を新たに搭載した。通信機能の追加により構成は複雑化したものの,外形寸法は約 35\,mm $\times$ 55\,mm に抑えられている。測定機器2と比較すると,外形面積は約 3500\,mm$^2$ から約 1925\,mm$^2$ へと減少しており,通信機能を追加しながらも,約 45\% の小型化が実現されている。

以上の結果から,本研究では測定機器1から測定機器4へと段階的に改良を行うことで,機能追加と小型化の両立を実現してきたといえる。特に,測定機器3および測定機器4では,
日常生活において携帯可能なサイズを達成しており,ネックレスや腰部への装着といった利用形態にも適したCO$_2$ 測定デバイスであると評価できる。これらの小型化に関する評価結果は,携帯型 CO$_2$ 測定デバイスとしての実用性を示す重要な要素であり,本研究で提案する測定機器の有効性を定量的に裏付ける結果である。

\section{アルコール、呼気への反応評価}

\begin{figure}[H]
\centering
\includegraphics[width=0.9\linewidth]{./figures/alcohol}
\caption{
アルコール、呼気への反応評価
}
\label{fig:height_co2_comparison}
\end{figure}







