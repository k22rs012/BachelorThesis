% 評価結果


% 消費電力:USBテスタ(AVHzYct-3)を使用して測定し、消費電力を確認。
% 安定性:波形データから安定性を評価し、機器の安定した計測・データ送信の有無を確認。
% 稼働時間:Webシステムにデータが表示されなくなるまでの時間を測定。
% 精度:市販のTesto 405iと比較して、開発したセンサの精度を確認。
% 寸法:各機器のサイズ比較。
% 価格:各機器のコストパフォーマンスを評価。
% 他社サービスとの比較:他社の製品やサービス(Testo 405i、Wi-Musu、Gill WindObserver 65など)との比較。




%\subsection{風速の測定精度}
%吹出し口用の測定機器の風速の測定精度を評価するため,405iと比較して風速を測定した.
%405iの測定結果を図\ref{fig:BlowChartTesto}に,吹出し口用測定機器の測定結果を図\ref{fig:BlowChartMe}に,
%測定値の平均と誤差率を表\ref{tab:BlowChart}に示す.
%測定結果を比較すると,
%タイプ2とタイプ3の設置方法2が405iと比較して風速が低いことが確認された.
%誤差率に関しても,タイプ2とタイプ3の設置方法2が高い誤差率を示している.
%しかしすべてのセンサが空調のパワーを変えたタイミングで風速が変化しているため,
%受け取った値に補正をかけることで,風速の測定精度を向上させることができるだろう.

%\begin{figure}[h]
 %\begin{minipage}{0.5\hsize}
  %\begin{center}
   %\includegraphics[width=0.9\linewidth]{./figures/BlowChartTesto.eps}
   %\caption{405iのグラフ}
   %\label{fig:BlowChartTesto}
  %\end{center}
 %\end{minipage}
 %\begin{minipage}{0.5\hsize}
  %\begin{center}
   %\includegraphics[width=0.9\linewidth]{./figures/BlowChartMe.eps}
   %\caption{吹出し口用測定機器のグラフ}
   %\label{fig:BlowChartMe}
  %\end{center}
 %\end{minipage}
%\end{figure}

%\begin{table}
%\centering
%\caption{風速の測定精度}
%\begin{tabular}{|c|r|r|}
%\hline
%\textbf{測定機器} & \textbf{平均風速(m/s)} & \textbf{誤差率(\%)} \\ \hline \hline
%405i & 5.4232 & 0\% \\ \hline
%吹き出し口タイプ2 & 4.1184 & 24.06\% \\ \hline
%吹き出し口タイプ3設置方法1 & 5.34816 & 1.38\% \\ \hline
%吹き出し口タイプ3設置方法2 & 3.18835 & 41.21\% \\ \hline
%\end{tabular}
%\label{tab:BlowChart}
%\end{table}


本章では5章で述べた測定方式で測定した結果と評価を述べる.



\section{概要}

\section{据え置き型センサとの比較結果}
  \subsection{CO$_2$ 濃度の時間変化}
  \subsection{高さ方向における測定結果}

\section{携帯時の測定結果}
  \subsection{装着位置による測定結果}
  \subsection{登山中の CO$_2$ 濃度変化}

\section{通信機能の動作結果}

\section{省電力性能の評価結果}









