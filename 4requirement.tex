% 要求仕様

\section{概要}
本章では,本研究で開発する小型 CO$_2$ 測定デバイスの仕様設計について述べる。
本研究の目的は,携帯可能でありながら高精度な CO$_2$ 濃度測定を可能とするデバイスを
試作し,日常生活のさまざまな環境における換気状況を把握できる仕組みを構築することである。
そのため,本デバイスには小型化,省電力化,測定精度,通信機能といった複数の要件が求められる。


%%%%%

\section{設計仕様}
本研究で開発する CO$_2$ 測定デバイスは,以下の仕様を満たすことを目標とする。

\begin{enumerate}
  \item 測定値と実際の CO$_2$ 濃度との差が小さいこと
  \item 様々な環境において CO$_2$ 濃度を測定できること
  \item 屋外環境を含む様々な場所で長時間利用できること
  \item 小型で持ち運び可能であること
  \item 測定した CO$_2$ 濃度を一定時間おきにサーバへアップロードできること
\end{enumerate}
\vskip1cm

仕様1では,測定値が実際の CO$_2$ 濃度を正確に反映していることを重視する。そのため,経済産業省および産業用ガス検知警報器工業会により制定された「二酸化炭素濃度測定器の選定等に関するガイドライン」を参考にする。同ガイドラインでは,適切な CO$_2$ 濃度測定器の条件として,検知原理に光学式(NDIR 方式など)を用いていること,補正用の機能が測定器に付帯していることが示されている。また,屋外環境において測定を行った際に,測定値が外気中の CO$_2$ 濃度(415~450 ppm 程度)に近い値を示すことや,測定器に呼気を吹きかけた際に CO$_2$ 濃度が大きく上昇すること,さらに,消毒用アルコールを塗布した手や布を測定器に近づけてもCO$_2$ 濃度の測定値が大きく変化しないことが求められている。本研究では,これらの条件に準拠した測定デバイスを設計・開発する。

仕様2では,据え置き型の CO$_2$ 測定器にはない特長として,利用者が移動しながら使用できる点を重視する。自宅や大学,研究室といった固定的な環境に加え,車内や飲食店など多様な環境においても問題なく測定できることを目標とし,携帯型測定デバイスとしての有効性を検証する。

仕様3および仕様4では,携帯型 CO$_2$ 測定デバイスとして屋外環境を含む様々な場所で長時間利用できることを重視する。そのため,商用電源や大型のモバイルバッテリに依存しない構成とし,小型のリチウムポリマーバッテリによる駆動を採用した。また,限られたバッテリ容量で長時間動作を実現するため,デバイス全体の省電力化を図るとともに,筐体の小型化を行った。これにより,ネックレスや腰にぶら下げて持ち運ぶことが可能となり,日常生活の中で手軽に使用できる測定デバイスの実現を目指す。

仕様5では,測定した CO$_2$ 濃度を一定時間間隔でサーバへアップロードする機能を持たせる。


%%%%%


