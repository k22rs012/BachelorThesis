% 設計


\section{設計の概要}
本章では,本研究で試作した携帯型 CO$_2$ 測定デバイスの設計について述べる.本研究では,日常生活のさまざまな環境において,利用者がデバイスを携帯しながらCO$_2$ 濃度を柔軟に測定できることを目的とし,小型化,省電力化,測定精度,および通信機能を考慮した設計を行った.

従来の CO$_2$ 測定機器は,据え置き型として室内環境の評価に用いられることが多く,設置位置が固定されるため,利用者の行動や移動に伴う空気環境の変化を直接的に把握することは困難であった.これに対し本研究では,利用者自身が測定機器を携帯することで,生活空間や移動環境におけるCO$_2$ 濃度の変化を把握できる測定システムの実現を目指した.

本デバイスは,CO$_2$ センサ,マイクロコントローラ,通信モジュール,および電源系から構成されており,測定データを一定時間間隔で自動的に取得する方式に加え,利用者がボタンを操作することで任意のタイミングで測定を行う方式の両方に対応した構成としている.取得した測定データは,屋内外を含むさまざまな環境からサーバへ送信されることを想定している.

このように,定期的な測定による環境変化の把握と,利用者の判断による即時的な測定を両立することで,携帯型 CO$_2$ 測定デバイスとしての実用性および柔軟性の向上を図った.

本章では,これらの目的を達成するために採用した設計方針,システム全体の構成,動作および通信の考え方,ならびに省電力化に関する設計上の考慮点について順に説明する.

\section{設計方針}
本研究における携帯型 CO$_2$ 測定デバイスの設計にあたっては,経済産業省および産業用ガス検知警報器工業会により制定された「二酸化炭素濃度測定器の選定等に関するガイドライン」を参考とした.CO$_2$濃度測定器に求められる基本的な性能として,以下の点が示されている.
\begin{itemize}
  \item CO$_2$の検知原理として光学式を用いていること
  \item 測定値の安定性を確保するための補正機能を有すること
  \item 屋外環境におけるCO$_2$濃度が415~450\,ppm程度を示すこと
  \item 呼気を吹きかけた際に測定値が大きく増加すること
  \item 消毒用アルコールを近づけた場合に測定値が大きく変化しないこと
\end{itemize}

本研究では,これらの要件を満たすことを設計上の前提条件とし,センサの選定および測定デバイスの設計を行った.これらの指針を踏まえ,日常生活環境において利用者が携帯して使用する際にも,CO$_2$ 濃度を安定して測定できるデバイスの実現を目指し,センサの選定およびシステム設計を行った.その上で,携帯型 CO$_2$ 測定デバイスとして,利用者が日常生活の中で無理なく使用できることを重視し,小型化,省電力化,測定精度,および通信機能のバランスを考慮した設計方針を採用した.特に,小型化,省電力化,測定精度,および通信機能のバランスを考慮し,実用性を意識した設計を行った.まず,小型化については,据え置き型測定器とは異なり,利用者が身につけて使用することを想定し,筐体サイズおよび重量を可能な限り抑える設計とした.これにより,屋内外を問わず携帯可能な測定デバイスの実現を目指した.次に,省電力化については,商用電源に依存せずバッテリ駆動での長時間利用を前提にした設計とした.測定および通信処理に伴う消費電力を抑えることを重視し,必要な処理のみを周期的に実行する構成を想定した.測定精度に関しては,室内外の換気状態を適切に評価するため,CO$_2$ 濃度を安定して測定できることを重視した.そのため,測定原理や補正機能を考慮し,環境変化の影響を受けにくい測定が可能となる設計とした.通信機能については,常時通信を行うのではなく,一定時間間隔で測定データを送信する方式を想定した.これにより,通信回数を抑え,省電力化を図る設計とした.

\section{システム全体構成}
本研究で設計した携帯型 CO$_2$ 測定デバイスは,CO$_2$ センサ,マイクロコントローラ,通信モジュール,および電源系から構成される.本節では,これらの構成要素の役割と,データおよび電力の流れについて説明する.

CO$_2$ センサは,周囲環境中の CO$_2$ 濃度を測定し,その測定値をマイクロコントローラへ出力する.マイクロコントローラは,測定データの取得および管理を行うとともに,通信制御や省電力制御を担う中核的な役割を果たす.

本デバイスでは,測定方式として一定時間間隔で自動的に測定を行う周期測定と,利用者がボタンを操作することで任意のタイミングで測定を行う任意測定の二つの方式を採用した.周期測定により,時間的な CO$_2$ 濃度変化を継続的に把握することが可能となり,任意測定により,利用者が特定の場所や状況において即時的に測定を行うことが可能となる.

取得された測定データは,通信モジュールを介してサーバへ送信されることを想定している.通信方式については,屋内外や移動環境での利用を考慮し,特定の設置環境に依存しない構成とした.これにより,測定場所に制約されることなく,継続的なデータ収集が可能となる.

電源系については,小型バッテリによる駆動を前提とし,測定および通信処理が行われていない時間帯には,各構成要素が待機状態へ移行する構成とした.このような構成により,携帯性と省電力性を両立したCO$_2$ 測定システムの実現を目指した.

\section{動作設計}
本研究で設計した携帯型 CO$_2$ 測定デバイスは,周期的な動作と利用者操作による動作の両方に対応した構成とした.本節では,デバイス全体の基本的な動作の流れと,その設計上の考え方について述べる.

本デバイスは,起動後に周囲環境の CO$_2$ 濃度を測定し,取得した測定データを内部で保持した後,必要に応じて通信処理を行う.この一連の動作は,一定時間間隔で自動的に繰り返される周期動作として設計した.これにより,長時間にわたる環境変化を継続的に把握することが可能となる.

一方で,利用者が特定の場所や状況において即時的に CO$_2$ 濃度を確認したい場合に対応するため,ボタン操作による任意測定機能を実装した.ボタンが押下されると,周期測定とは独立して測定処理が実行され,その時点の CO$_2$ 濃度が取得される.この測定結果も,通常の測定データと同様に必要に応じて通信処理が行われる.

測定および通信処理が完了した後は,デバイス全体が待機状態へ移行する.この待機状態では,消費電力の大きい処理を停止することで,バッテリ消費を抑える設計とした.本研究では,常時動作による連続測定ではなく,測定・通信・待機を繰り返す周期動作を基本とし,必要に応じて任意測定を行う構成とすることで,省電力性と実用性の両立を図った.

\section{通信設計}
本研究では,携帯型 CO$_2$ 測定デバイスとしての利用を想定し,通信機能についても省電力性と実用性の両立を重視した設計とした.本節では,通信方式および通信タイミングに関する設計上の考え方を述べる.通信設計においては,測定データを常時送信する構成ではなく,一定時間間隔でまとめて送信する方式を想定した.これは,通信処理がデバイス全体の消費電力に与える影響が大きいことを考慮し,不要な通信回数を削減することを目的としている.また,本デバイスは屋外環境や移動中での利用を想定しているため,特定の通信環境に依存しない構成とした.これにより,自宅や大学といった屋内環境に限らず,車内や外出先においても測定データの送信が可能となる設計とした.

通信処理は,測定データの取得後に必要に応じて実行されるものとし,通信が不要な時間帯においては,通信機能を待機状態とする設計を採用した.このような通信設計により,携帯型デバイスとしての利便性を確保しつつ,バッテリ消費の低減を図ることを目指した.

\section{省電力設計}
携帯型 CO$_2$ 測定デバイスとして長時間利用を可能とするため,本研究では省電力性を重視した設計を行った.本節では,デバイス全体の消費電力を抑えるために考慮した設計上の方針について述べる.本デバイスはバッテリ駆動を前提としているため,常時動作による連続測定ではなく,測定・通信・待機を繰り返す周期動作を基本とした設計とした.これにより,不要な処理が実行される時間を最小限に抑え,全体の消費電力低減を図ることを目的とした.

省電力設計においては,特に通信処理が消費電力に与える影響が大きい点を考慮した.そのため,測定ごとに通信を行う構成は採用せず,一定時間ごとに測定データをまとめて送信する方式を想定した.これにより,通信回数の削減とバッテリ消費の低減を両立する設計とした.また,測定および通信処理が完了した後は,デバイス全体が待機状態となることを前提とし,動作していない時間帯の消費電力を抑える設計とした.このような省電力設計により,携帯型デバイスとしての実用性を確保しつつ,長時間の連続利用を可能とすることを目指した.


%%%%%
%%%%%


