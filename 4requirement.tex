% 設計


\section{設計の概要}
本章では,本研究で試作した携帯型 CO$_2$ 測定デバイスの設計について述べる。本研究では,日常生活のさまざまな環境において利用者が携帯しながらCO$_2$ 濃度を測定できるデバイスの実現を目的とし,小型化,省電力化,測定精度,および通信機能を考慮した設計を行った。本デバイスは,CO$_2$ センサ,マイクロコントローラ,通信モジュール,および電源系から構成されており,測定データを一定時間間隔で取得し,屋外環境を含むさまざまな場所からサーバへ送信することを想定している。

本章では,これらの目的を達成するために採用した設計方針,システム全体の構成,動作および通信の考え方,ならびに省電力化に関する設計上の考慮点について説明する。

\section{設計方針}
本研究では,携帯型 CO$_2$ 測定デバイスの実現を目的として,利用者が日常生活の中で無理なく使用できることを重視した設計方針を採用した。特に,小型化,省電力化,測定精度,および通信機能のバランスを考慮し,実用性を意識した設計を行った。

まず,小型化については,据え置き型測定器とは異なり,利用者が身につけて使用することを想定し,筐体サイズおよび重量を可能な限り抑える設計とした。これにより,屋内外を問わず携帯可能な測定デバイスの実現を目指した。

次に,省電力化については,商用電源に依存せずバッテリ駆動での長時間利用を前提とした設計とした。測定および通信処理に伴う消費電力を抑えることを重視し,必要な処理のみを周期的に実行する構成を想定した。

測定精度に関しては,室内外の換気状態を適切に評価するため,CO$_2$ 濃度を安定して測定できることを重視した。そのため,測定原理や補正機能を考慮し,環境変化の影響を受けにくい測定が可能となるよう設計した。

通信機能については,常時通信を行うのではなく,一定時間間隔で測定データを送信する方式を想定した。これにより,通信回数を抑え,省電力化と移動環境での利用の両立を図る設計とした。

\section{システム全体構成}
本研究で設計した携帯型 CO$_2$ 測定デバイスは,CO$_2$ センサ,マイクロコントローラ,通信モジュール,および電源系から構成される。本節では,これらの構成要素の関係性と,データおよび電力の流れについて説明する。

本システムでは,CO$_2$ センサによって周囲環境中の CO$_2$ 濃度を測定し,その測定データをマイクロコントローラで取得・処理する構成とした。マイクロコントローラは,測定データの管理および通信制御の中心的な役割を担い,一定時間間隔で通信モジュールを介して測定データをサーバへ送信することを想定している。通信機能については,屋外環境や移動中の利用を考慮し,Wi-Fi 環境に依存しない構成とした。これにより,測定場所に制約されることなく,継続的なデータ収集が可能となる。電源系については,小型バッテリによる駆動を前提とし,システム全体が低消費電力で動作する構成を想定した。測定および通信以外の時間は,各構成要素が待機状態となることで,バッテリ消費を抑える設計とした。

以上の構成により,携帯性と測定精度を両立しつつ,移動環境においても利用可能なCO$_2$ 測定システムを実現することを目指した。

\section{動作設計}
本研究で設計した携帯型 CO$_2$ 測定デバイスは,周期的な動作を前提とした構成とした。本節では,デバイス全体の基本的な動作の流れと,その設計上の考え方について述べる。

本デバイスは,起動後に周囲環境の CO$_2$ 濃度を測定し,取得した測定データを内部で保持した後,必要に応じて通信処理を行う動作を想定している。測定処理は一定時間間隔で繰り返し実行され,連続的な環境変化を把握できる構成とした。また,携帯型デバイスとしての利用を考慮し,測定および通信処理が完了した後は,待機状態へ移行する動作を前提とした設計とした。これにより,不要な処理を抑え,バッテリ消費を低減することを目指した。本研究では,常時動作による連続測定ではなく,測定・通信・待機を繰り返す周期動作を採用することで,省電力性と実用性の両立を図る設計とした。この動作設計に基づき,後述する通信設計および省電力設計を行っている。

\section{通信設計}
本研究では,携帯型 CO$_2$ 測定デバイスとしての利用を想定し,通信機能についても省電力性と実用性の両立を重視した設計とした。本節では,通信方式および通信タイミングに関する設計上の考え方を述べる。通信設計においては,測定データを常時送信する構成ではなく,一定時間間隔でまとめて送信する方式を想定した。これは,通信処理がデバイス全体の消費電力に与える影響が大きいことを考慮し,不要な通信回数を削減することを目的としている。また,本デバイスは屋外環境や移動中での利用を想定しているため,特定の通信環境に依存しない構成とした。これにより,自宅や大学といった屋内環境に限らず,車内や外出先においても測定データの送信が可能となる設計とした。

通信処理は,測定データの取得後に必要に応じて実行されるものとし,通信が不要な時間帯においては,通信機能を待機状態とする設計を採用した.このような通信設計により,携帯型デバイスとしての利便性を確保しつつ,バッテリ消費の低減を図ることを目指した。

\section{省電力設計}
携帯型 CO$_2$ 測定デバイスとして長時間利用を可能とするため,本研究では省電力性を重視した設計を行った。本節では,デバイス全体の消費電力を抑えるために考慮した設計上の方針について述べる。本デバイスはバッテリ駆動を前提としているため,常時動作による連続測定ではなく,測定・通信・待機を繰り返す周期動作を基本とした設計とした。これにより,不要な処理が実行される時間を最小限に抑え,全体の消費電力低減を図ることを目的とした。

省電力設計においては,特に通信処理が消費電力に与える影響が大きい点を考慮した。そのため,測定ごとに通信を行う構成は採用せず,一定時間ごとに測定データをまとめて送信する方式を想定した。これにより,通信回数の削減とバッテリ消費の低減を両立する設計とした。また,測定および通信処理が完了した後は,デバイス全体が待機状態となることを前提とし,動作していない時間帯の消費電力を抑える設計とした。このような省電力設計により,携帯型デバイスとしての実用性を確保しつつ,長時間の連続利用を可能とすることを目指した。


%%%%%

\begin{comment}
\section{設計仕様}
本研究で開発する CO$_2$ 測定デバイスは,以下の仕様を満たすことを目標とする。

\begin{enumerate}
  \item 測定値と実際の CO$_2$ 濃度との差が小さいこと
  \item 様々な環境において CO$_2$ 濃度を測定できること
  \item 屋外環境を含む様々な場所で長時間利用できること
  \item 小型で持ち運び可能であること
  \item 測定した CO$_2$ 濃度を一定時間おきにサーバへアップロードできること
\end{enumerate}
\vskip1cm

仕様1では,測定値が実際の CO$_2$ 濃度を正確に反映していることを重視する。そのため,経済産業省および産業用ガス検知警報器工業会により制定された「二酸化炭素濃度測定器の選定等に関するガイドライン」を参考にする。同ガイドラインでは,適切な CO$_2$ 濃度測定器の条件として,検知原理に光学式(NDIR 方式など)を用いていること,補正用の機能が測定器に付帯していることが示されている。また,屋外環境において測定を行った際に,測定値が外気中の CO$_2$ 濃度(415~450 ppm 程度)に近い値を示すことや,測定器に呼気を吹きかけた際に CO$_2$ 濃度が大きく上昇すること,さらに,消毒用アルコールを塗布した手や布を測定器に近づけてもCO$_2$ 濃度の測定値が大きく変化しないことが求められている。本研究では,これらの条件に準拠した測定デバイスを設計・開発する。

仕様2では,据え置き型の CO$_2$ 測定器にはない特長として,利用者が移動しながら使用できる点を重視する。自宅や大学,研究室といった固定的な環境に加え,車内や飲食店など多様な環境においても問題なく測定できることを目標とし,携帯型測定デバイスとしての有効性を検証する。

仕様3および仕様4では,携帯型 CO$_2$ 測定デバイスとして屋外環境を含む様々な場所で長時間利用できることを重視する。そのため,商用電源や大型のモバイルバッテリに依存しない構成とし,小型のリチウムポリマーバッテリによる駆動を採用した。また,限られたバッテリ容量で長時間動作を実現するため,デバイス全体の省電力化を図るとともに,筐体の小型化を行った。これにより,ネックレスや腰にぶら下げて持ち運ぶことが可能となり,日常生活の中で手軽に使用できる測定デバイスの実現を目指す。

仕様5では,測定した CO$_2$ 濃度を一定時間間隔でサーバへアップロードする機能を持たせる。
\end{comment}

%%%%%


