% 要求仕様

\section{概要}
本章では,本研究で開発する小型 CO$_2$ 測定デバイスの仕様設計について述べる。
本研究の目的は,携帯可能でありながら高精度な CO$_2$ 濃度測定を可能とするデバイスを
試作し,日常生活のさまざまな環境における換気状況を把握できる仕組みを構築することである。
そのため,本デバイスには小型化,省電力化,測定精度,通信機能といった複数の要件が求められる。


%%%%%

\section{設計仕様}
本研究で開発する CO$_2$ 測定デバイスは,以下の仕様を満たすことを目標とする。

\begin{enumerate}
  \item 測定値と実際の CO$_2$ 濃度との差が小さいこと
  \item 様々な環境において CO$_2$ 濃度を測定できること
  \item リチウムポリマーバッテリにより駆動できること
  \item 小型で持ち運び可能であること
  \item 測定した CO$_2$ 濃度を一定時間おきにサーバへアップロードできること
\end{enumerate}

\vskip1cm

仕様1では,測定値が実際の CO$_2$ 濃度を正確に反映していることを重視する。そのため,経済産業省および産業用ガス検知警報器工業会により制定された「二酸化炭素濃度測定器の選定等に関するガイドライン」を参考にする。同ガイドラインでは,適切な CO$_2$ 濃度測定器の条件として,検知原理に光学式(NDIR 方式など)を用いていること,補正用の機能が測定器に付帯していることが示されている。また,屋外環境において測定を行った際に,測定値が外気中の CO$_2$ 濃度(415~450 ppm 程度)に近い値を示すことや,測定器に呼気を吹きかけた際に CO$_2$ 濃度が大きく上昇すること,さらに,消毒用アルコールを塗布した手や布を測定器に近づけてもCO$_2$ 濃度の測定値が大きく変化しないことが求められている。本研究では,これらの条件に準拠した測定デバイスを設計・開発する。

仕様2では,据え置き型の CO$_2$ 測定器にはない特長として,利用者が移動しながら使用できる点を重視する。自宅や大学,研究室といった固定的な環境に加え,車内や飲食店など多様な環境においても問題なく測定できることを目標とし,携帯型測定デバイスとしての有効性を検証する。

仕様3および仕様4では,携帯型デバイスとしての実用性を高めるため,電源方式および筐体サイズに着目する。本デバイスはリチウムポリマーバッテリによる駆動を可能とし,商用電源に依存せず長時間動作できる構成とする。また,デバイス全体を小型化することで,バッグやポケットに収納して持ち運ぶことができ,日常生活の中で手軽に使用できる測定器の実現を目指す。

仕様5では,測定した CO$_2$ 濃度を一定時間間隔でサーバへアップロードする機能を持たせる。これにより,測定データを時系列で蓄積し,
後から測定環境ごとの変化を比較・分析することが可能となる。


%%%%%

\subsection{測定値の補正}
本測定機器に搭載された風速センサは一方向のみ測定可能であり,測定角度が狭いという制約がある.
そのため,空調のように吹き出し方向が比較的固定されている測定対象であっても,一回の測定では正確な結果を得ることが難しい場合がある.
これを改善するため,オーバーサンプリングを実施し,1秒間に8回の測定を行い,その平均値および最大値を測定結果として採用することで,測定値の誤差を低減している.

さらに,測定値を現在メンテナンスで使用している風速計の測定値に近づけるため,空調の強度ごとに手動でキャリブレーションを行った.

温湿度センサについては,使用しているライブラリにオーバーサンプリングおよびIIRフィルタを指定する機能がある.
これを利用し,温度については16回,湿度と気圧については各8回のサンプリングを行った.
IIRフィルタは使用すると温度の変化が滑らかになり,ノイズからの影響に強くなるが,温度の急激な変化への対応に遅延が生じるため,今回は使用しなかった.



\section{測定機器の稼働時間延長の検討}
本節では,測定機器の稼働時間を延長する方法について検討する.
現状では,どちらの測定機器も約23時間の稼働が可能であり,1日内の測定には対応できるが,1週間以上の連続稼働は困難な状況である.
このため,現在の処理手順や消費電力を分析し,削減を試みた.

電力を削減する方法として,ディープスリープとデータの一括送信を行うことを検討した.
ディープスリープは,マイコンの電力消費を抑えるための機能であり,マイコンをスリープ状態に移行させることで,消費電力を抑えることができる.
データの一括送信は,データを一定時間間隔でサーバに送信するのではなく,一定時間間隔でデータを取得し,一定回数分のデータをまとめてサーバに送信することで,通信回数を減らし,消費電力を抑えることができる.
データの一括送信は,消費電力を減らすことは可能だが,目的であるリアルタイムでの測定値の閲覧が困難になるため,今回はディープスリープを採用した.




\section{測定値閲覧機能(Webシステム)}
本項では,Webシステムの測定値閲覧機能を実現する際のクライアント側の設計について説明する.
測定値閲覧機能では,要求仕様に対応するために,主に3つの機能を設計した.
一つは測定機器ごとの測定値を閲覧する機能,
二つ目はグループごとの測定値を閲覧する機能,
三つ目は測定機器に名前を付け,グループ分けする機能である.

測定値を閲覧する機能は20秒以内にデータを更新し,リアルタイムでの閲覧を可能にする.
グループ分けの機能は測定値を記録するデータベースから現在存在する測定機器のIDを取得し,
ID,名前,所属グループ名を表示できるものとする.
測定機器がまだ登録されていない場合は,空欄を表示し,新たに登録することができる.



\subsection{測定値閲覧機能(サーバ)}
本項では,Webシステムの測定値閲覧機能を実現する際のサーバ側の処理について説明する.サーバ側は本研究室のKSU アプリやiOS 版大富アプリなどで使用しているサーバを使用している.
使用しているテーブルを表\ref{fig:tbl.airflow.esp32c6},表\ref{tbl:vacuum_ac},表\ref{tbl:blow.device},表\ref{tbl:vacuum.device},表\ref{tbl:group}に示す.
空調の吹出し口側の測定機器が測定した値はtbl.airflow.esp32c6に,空調の吸込み口側の測定機器が測定した値はtbl.vacuum.acに格納される.
それぞれのデバイスIdやデバイス名,所属しているグループなどをtbl.blow.device,tbl.vacuum.device,tbl.groupに格納している.
全てのテーブルに共通してid,created,modified,flagの4つのカラムを定義している.
順番に主キー,作成日の保存,最終更新日の保存,現在その行のデータが有効かどうかの削除フラグとなっている.
それ以外の定義内容として,tbl.airflow.esp32c6では風速,温度,温度差,湿度,気圧のカラムがある.
風速のカラムとしてrawAirFlowData,metersPSAirFlowData,milesPHAirFlowData,maxRawAirFlowData,maxMetersPSAirFlowData,maxMilesPHAirFlowDataの6つを定義している.
それぞれ風速センサの生データ,メートル毎秒に変換した風速,マイル毎時に変換した風速を格納するもので,Maxがついているものが最大値,ついていないものが平均値を格納している.
最大値と平均値を算出するために風速は一回の測定に八回センサから値を読み取っている.
そのほかのgasはBME680から受け取れるガス(有機溶剤,アルコール等)の値を格納し,
temperatureで温度,humidityで湿度,pressureで気圧を格納している.
tempertureDiffはtemperatureの温度とtbl.vacuum.acに格納された吸込み口の温度の差を格納している.

tbl.vacuum.acでは吸込み口で測定する測定機器にはBME680のみを接続しているため,
gas,temperature,humidity,pressureの四つが定義されている.

tbl.device.pairingではデバイスのID,名前,グループIDを保存するdeviceId,deviceName,groupIdの3つが定義されている.



